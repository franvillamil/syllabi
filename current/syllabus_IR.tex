\documentclass[12pt, a4paper]{article}
\usepackage[margin = 2cm]{geometry}
\usepackage{graphicx}
\usepackage[english]{babel}
\usepackage[utf8]{inputenc}

\usepackage{color}
\definecolor{dark_blue}{rgb}{0.2, 0.0, .7}
\usepackage[colorlinks = TRUE,
			allcolors = dark_blue]{hyperref}

\usepackage{setspace}
\setstretch{1.25}
\renewcommand*\rmdefault{ppl}


\usepackage[]{titlesec}
    \titleformat*{\section}{\large\bf}
    \titleformat*{\subsection}{\normalsize\bf}
    \titleformat*{\subsubsection}{\normalsize\it}

%%%%%%%%%%%%%%%%%%%%%%%%%%%%%%%%%%%%
\begin{document}
\begin{center}
{\LARGE\bf International Relations}\\\vspace{10pt}
Master in Social Sciences\\Carlos III-Juan March Institute\\
%\\Universidad Carlos III de Madrid\\
\vspace{10pt}
{\large Spring 2025}\\
\end{center}

\vspace{15pt}

\begin{minipage}{0.6\textwidth}
\textbf{Francisco Villamil}\\
Email: \href{francisco.villamil@uc3m.es}{francisco.villamil@uc3m.es}\\
Office: 18.2.D15 (by appointment)
\end{minipage}\hfill
\begin{minipage}{0.39\textwidth}
\centering
\textbf{Time and place:}\\
Thursdays, 10h--13h\\Room 18.1.A01
\end{minipage}


% \vspace{10pt}
\section{Description}

This is a graduate-level course on contemporary research in International Relations. The core of the field focuses on the causes, dynamics, and consequences of aggression and cooperation between states, and has traditionally revolved around the study of war and peace.
We will focus on the study of international conflict, but also on related topics that are often considered within IR, such as the study of internal conflict and processes of political violence, the nature of state size and borders, the influence of the international system on domestic institutions, or processes of international diffusion of norms.
% Although, especially during the first sessions, we will review some classic works that have a strong theoretical component,
We will focus on recent research and on the methodological aspect of studying these topics empirically.

\section{Requirements and grading}

Each session will consist of a brief lecture introducing the topic and the main concepts, followed by a seminar discussion on the required readings. Students are expected to attend every week, read the required readings in advance, and actively participate in the discussion.
In addition, each student will have to present one of the complementary readings once during the semester.
The last session of the course will be a workshop where students present ideas and ongoing work for the final research paper.

\paragraph{Participation (30\%):} Students are expected to participation in all seminar discussions.%In addition, they have to choose three sessions and prepare a brief memo (around 2 pages) for each of them, summarizing and connecting readings. Submission before class starts.

\vspace{-5pt}\paragraph{Presentation (10\%):} Once during the semester, each student will present one of the complementary readings in class, discussing how it relates and complements the main readings.

% \vspace{-5pt}\paragraph{Workshop (10\%):} Students are expected to present and participate in the final workshop during the last session of the course.

\vspace{-5pt}\paragraph{Research paper (60\%):} Students are expected to write a short empirical research paper (maximum 4,000 words) in one of the topics covered in the course, using empirical evidence (qualitative, quantitative, or both) to answer a research question. A proposal will be presented and discussed in the last session in class, participation in the workshop will also be taken into account. Deadline TBD (around late May).

\newpage
\section{Course outline}

 % \hrule % ============================================================

\begin{table*}[!ht]
  \centering
  \setstretch{1.4}

  \begin{tabular}{lll}
    \textbf{Session 1} & January 30 & Introduction \\
    \textbf{Session 2} & February 6 & Interstate war \\
    \textbf{Session 3} & February 13 & State-building, nationalism, and war \\
    \textbf{Session 4} & February 20 & Democracy, war, and peace \\
    \textbf{Session 5} & February 27 & Civil wars \\
    \textbf{Session 6} & March 6 & Political economy \\
    \textbf{Session 7} & March 13 & Disaggregating and predicting conflict \\
    \textbf{Session 8} & March 20 & Wartime violence \\
    \textbf{Session 9} & March 27 & Terrorism \\
    \textbf{Session 10} & April 3 & Legacies of conflict \\
    \textbf{Session 11} & April 10 & Constructivism, norm diffusion, and TJ \\
    % \textit{No class} & April 17 & \textit{(Easter week)} \\
    \textit{No class} & April 24 & \\
    \textbf{Session 12} & May 8 & Research workshop \\
  \end{tabular}
\end{table*}

% \vspace{20pt}
% \hrule

\vspace{20pt}\noindent\textit{\textbf{Note:}} Some readings might change, I will notify well in advance.

\subsection*{Session 1: Introduction}

\noindent\textit{Complementary:}

\begin{itemize}
  \item Stephen M. Walt (1998) \href{https://doi.org/10.2307/1149275}{International Relations: One World, Many Theories.} \textit{Foreign Policy} 110: 29--32, 34--46.
  \item David A Lake (2011) \href{https://doi.org/10.1111/j.1468-2478.2011.00661.x}{Why `isms' Are Evil: Theory, Epistemology, and Academic Sects as Impediments to Understanding and Progress.} \textit{International Studies Quarterly} 55(2): 465--480.
  \item John J. Mearsheimer and Stephen M. Walt (2013) \href{https://doi.org/10.1177/1354066113494320}{Leaving theory behind: Why simplistic hypothesis testing is bad for International Relations.} \textit{European Journal of International Relations} 19(3): 427--457.
  % \item Bruce Bueno de Mesquita (1985) \href{https://doi.org/10.2307/2600500}{Toward a Scientific Understanding of International Conflict: A Personal View.} \textit{International Studies Quarterly} 29(2): 121--136.
	% \item Dan Reiter (2015) \href{https://doi.org/10.1146/annurev-polisci-053013-041156}{Should We Leave Behind the Subfield of International Relations?} \textit{Annual Review of Political Science} 18: 481--499.
\end{itemize}

% - DOMESTIC POLITICS, FOREIGN POLICY, AND THEORIES OF INTERNATIONAL RELATIONS
% - Sikking on constructivism?

\vspace{20pt}
\hrule
% ------------------------------------------------------------
\subsection*{Session 2: Interstate war}

\noindent\textit{Required:}

\begin{itemize}
  \item Kenneth N. Waltz (1959) \href{https://cup.columbia.edu/book/man-the-state-and-war/9780231188043}{\textit{Man, the State, and War: A Theoretical Analysis}}. New York: Columbia University Press. [Chapters 2, 4, and 6.]
  \item Paul Huth, Christopher Gelpi, and D. Scott Bennett (1993) \href{https://doi.org/10.2307/2938739}{The Escalation of Great Power Militarized Disputes: Testing Rational Deterrence Theory and Structural Realism.} \textit{American Political Science Review} 87(3): 609--623.
\end{itemize}

\noindent\textit{Complementary:}

\begin{itemize}
  \item James D. Fearon (1995) \href{https://doi.org/10.1017/S0020818300033324}{Rationalist Explanations for War.} \textit{International Organization} 49(3): 379--414.
  \item Kalevi J. Holsti (1991) \href{https://www.cambridge.org/core/books/peace-and-war/37AEB58913E6EF0834D40EFE086D32FE}{\textit{Peace and War: Armed Conflicts and International Order, 1648-1989.}} New York: Cambridge University Press. [esp. Chapter 1, pp. 1--24.]
  % \item Robert Powell (1996) \href{https://doi.org/10.2307/2945840}{Uncertainty, Shifting Power, and Appeasement.} \textit{American Political Science Review} 90(4): 749--764.
  % \item James D. Fearon (1994) \href{https://doi.org/10.2307/2944796}{Domestic Political Audiences and the Escalation of International Disputes.} \textit{American Political Science Review} 88(3): 577--592.
	\item Brett Ashley Leeds (2003) \href{https://doi.org/10.1111/1540-5907.00031}{Do Alliances Deter Aggression? The Influence of Military Alliances on the Initiation of Militarized Interstate Disputes}. \textit{American Journal of Political Science} 47(3): 427--439.
\end{itemize}

\vspace{20pt}
\hrule
% ------------------------------------------------------------
\subsection*{Session 3: State-building, nationalism, and war}

\noindent\textit{Required:}

\begin{itemize}
	\item Sascha O. Becker, Andreas Ferrara, Eric Melander, and Luigi Pascali (2020) \href{https://cepr.org/publications/dp15601}{Wars, Taxation and Representation: Evidence from Five Centuries of German History}. \textit{CEPR Discussion Paper} No. 15601.
	\item Andreas Wimmer and Brian Min (2006) \href{https://doi.org/10.1177/000312240607100601}{From Empire to Nation-State: Explaining Wars in the Modern World, 1816–2001}. \textit{American Sociological Review} 71: 867--897.
  \item Agustina S. Paglayan (2022) \href{https://doi.org/10.1017/S0003055422000247}{Education or Indoctrination? The Violent Origins of Public School Systems in an Era of State-Building.} \textit{American Political Science Review} 116(4): 1242--1257.
\end{itemize}

\noindent\textit{Complementary:}

\begin{itemize}
  % \item Charles Tilly (1985) `War making and state making as organized crime.' In \href{https://bibliotecas.uc3m.es/permalink/f/1qk6at5/34UC3M_ALMA21176158990004213}{\textit{Bringing the State Back In}} (ed. P. Evans, D. Rueschemyer \& T. Skocpol), New York: Cambridge University Press, 169--187.
  \item Carl Müller-Crepon, Guy Schvitz, and Lars-Erik Cederman (2024) \href{https://doi.org/10.1177/00220027241227897}{``Right-Peopling'' the State: Nationalism, Historical Legacies, and Ethnic Cleansing in Europe, 1886-2020.} \textit{Journal of Conflict Resolution}, online first.
	\item Mark Mazower (2002) \href{https://doi.org/10.1086/ahr/107.4.1158}{Violence and the State in the Twentieth Century}. \textit{American Historical Review} 107(4): 1158--1178.
	% \item Lars-Erik Cederman, T.	Camber Warren, and Didier Sornette (2011) \href{https://doi.org/10.1017/S0020818311000245}{Testing Clausewitz: Nationalism, Mass Mobilization, and the Severity of War}. \textit{International Organization} 65(4): 605--638.
  % \item Barry R. Posen (1993) \href{https://doi.org/10.2307/2539098}{Nationalism, the Mass Army, and Military Power}. \textit{International Security} 18(2): 80--124.
\end{itemize}

\vspace{20pt}
\hrule
% ------------------------------------------------------------
\subsection*{Session 4: Democracy, war, and peace}

% NOTE: other options to consider & replace:

% Rosato, Sebastian. 2003. “The Flawed Logic of Democratic Peace Theory.” American Political Science Review, 97(4): 585-602.
% Mousseau, Michael. 2009. “The Social Market Roots of Democratic Peace.” International Security, 33(4): 52-86.
% Dafoe, Allan. 2011. “Statistical Critiques of the Democratic Peace: Caveat Emptor.” American Journal of Political Science, 55(2): 247-262.

\noindent\textit{Required:}

\begin{itemize}
	\item William Dixon (1994) \href{https://doi.org/10.2307/2944879}{Democracy and the Peaceful Settlement of Conflicts}. \textit{American Political Science Review} 88(1): 14--32.
  \item Lars-Erik Cederman (2001) \href{https://doi.org/10.1017/S0003055401000028}{Back to Kant: Reinterpreting the Democratic Peace as a Macrohistorical Learning Process.} \textit{American Political Science Review} 95(1): 15--31.
	\item Kristian Skrede Gleditsch and Michael D. Ward (2006) \href{https://doi.org/10.1017/S0020818306060309}{Diffusion and the International Context of Democratization}. \textit{International Organization} 60(4): 911--933.
	\end{itemize}

\noindent\textit{Complementary:}

\begin{itemize}
	\item Christopher Layne (1994) \href{https://doi.org/10.2307/2539195}{Kant or Cant: The Myth of the Democratic Peace}. \textit{International Security} 19(2): 5--49.
  \item Michael D. Ward and Kristian S. Gleditsch (1998) \href{https://doi.org/10.2307/2585928}{Democratizing for Peace}. \textit{American Political Science Review} 92(1): 51--61.
	% \item Sara McLaughlin Mitchell (2002) \href{https://doi.org/10.2307/3088431}{A Kantian System? Democracy and Third Party Conflict Resolution.} \textit{American Journal of Political Science} 46(4): 749-759.
	\item John R. Oneal, Bruce Russett, and Michael L. Berbaum (2003) \href{https://doi.org/10.1111/1468-2478.4703004}{Causes of Peace: Democracy, Interdependence, and International Organizations, 1885–1992}. \textit{International Studies Quarterly} 47(3): 371--393.
	\item Jon Pevehouse and Bruce Russett (2006) \href{https://doi.org/10.1017/S0020818306060322}{Democratic International Governmental Organizations Promote Peace}. \textit{International Organization} 60(4): 969--1000.
  % \item Michael D. Ward, Randolph M. Siverson, and Xun Cao (2007) \href{https://doi.org/10.1111/j.1540-5907.2007.00269.x}{Disputes, Democracies, and Dependencies: A Reexamination of the Kantian Peace}. \textit{American Journal of Political Science} 51(3): 583--601.
\end{itemize}

% War and peace in space and time: The role of democratization Kristian S Gleditsch, Michael D Ward

\vspace{20pt}
\hrule
% ------------------------------------------------------------
\subsection*{Session 5: Civil wars}

\noindent\textit{Required:}

\begin{itemize}
  \item James D. Fearon and David Laitin (2003) \href{https://doi.org/10.1017/S0003055403000534}{Ethnicity, Insurgency, and Civil War} \textit{American Political Science Review} 97(1): 75--90.
  \item Lars-Erik Cederman, Andreas Wimmer, and Brian Min (2010) \href{https://doi.org/10.1017/S0043887109990219}{Why Do Ethnic Groups Rebel? New Data and Analysis.} \textit{World Politics} 62(1): 87--119.
\end{itemize}

\vspace{10pt}

\noindent\textit{Complementary:}

\begin{itemize}
  \item Michael Doyle and Nicholas Sambanis (2000) \href{https://doi.org/10.2307/2586208}{International Peacebuilding: A Theoretical and Quantitative Analysis.} \textit{American Political Science Review} 94(4): 779--802.
  \item Barry Posen (1993) \href{https://doi.org/10.1080/00396339308442672}{The Security Dilemma and Ethnic Conflict.} \textit{Survival} 35:1: 27-47.
  \item Lars-Erik Cederman, Kristian Skrede Gleditsch, and Nils B. Weidmann (2011) \href{https://doi.org/10.1017/S0003055411000207}{Horizontal Inequalities and Ethnonationalist Civil War: A Global Comparison.} \textit{American Political Science Review} 105(3): 478--495.
  % \item Lars-Erik Cederman and Luc Girardin (2007) \href{https://doi.org/10.1017/S0003055407070086}{Beyond Fractionalization: Mapping Ethnicity onto Nationalist Insurgencies.} \textit{American Political Science Review} 101(1): 173--185.
  % \item Michael D. Ward, Nils W. Metternich, Cassy L. Dorff, Max Gallop, Florian M. Hollenbach, Anna Schultz, and Simon Weschle (2013) \href{https://doi.org/10.1111/misr.12072}{Learning from the Past and Stepping into the Future: Toward a New Generation of Conflict Prediction.} \textit{International Studies Review} 15(4): 473--490.
  \item Stathis Kalyvas and Laia Balcells (2010) \href{https://doi.org/10.1017/S0003055410000286}{International System and Technologies of Rebellion: How the End of the Cold War Shaped Internal Conflict}. \textit{American Political Science Review} 104(3): 415--429.
	\item Chaim Kaufmann (1996) \href{https://doi.org/10.1162/isec.20.4.136}{Possible and Impossible Solutions to Ethnic Civil Wars.} \textit{International Security} 20:4: 136-175.
	% \item Nicholas Sambanis (2004) \href{https://doi.org/10.1017/S1537592704040149}{Using Case Studies to Expand Economic Models of Civil War}. \textit{Perspectives on Politics} 2(2):259--279.
\end{itemize}


\vspace{20pt}
\hrule
% ------------------------------------------------------------
\subsection*{Session 6: Political economy}

\noindent\textit{Required:}

\begin{itemize}
  \item Erik Gartzke (2007) \href{https://doi.org/10.1111/j.1540-5907.2007.00244.x}{The Capitalist Peace}. \textit{American Journal of Political Science} 51(1): 166--191.
	\item Vally Koubi, Gabriele Spilker, Tobias Böhmelt, and Thomas Bernauer (2014) \href{https://doi.org/10.1177/002234331349345}{Do natural resources matter for interstate and intrastate armed conflict?} \textit{Journal of Peace Research} 51(2): 227--243.
	\item Nicolas Berman, Mathieu Couttenier, Dominic Rohner, and Mathias Thoenig (2017) \href{https://doi.org/10.1257/aer.20150774}{This Mine is Mine! How Minerals Fuel Conflicts in Africa}. \textit{American Economic Review} 107(6): 1564--1610.
\end{itemize}

\noindent\textit{Complementary:}

\begin{itemize}
  \item Edward Miguel, Shanker Satyanath, and Ernest Sergenti (2004) \href{https://doi.org/10.1086/421174}{Economic Shocks and Civil Conflict: An Instrumental Variables Approach}. \textit{Journal of Political Economy} 112(4): 725--753.
	\item Oeindrila Dube and Juan F. Vargas (2013) \href{https://doi.org/10.1093/restud/rdt009}{Commodity Price Shocks and Civil Conflict: Evidence from Colombia}. \textit{The Review of Economic Studies} 80(4): 1384--1421.
	\item Graeme Blair, Darin Christensen, and Aaron Rudkin (2021) \href{https://doi.org/10.1017/S0003055420000957}{Do Commodity Price Shocks Cause Armed Conflict? A Meta-Analysis of Natural Experiments}. \textit{American Political Science Review} 115(2): 709--716.
	\item Philip Hunziker and Lars-Erik Cederman (2017) \href{https://doi.org/10.1177/0022343316687365}{No extraction without representation: The ethno-regional oil curse and secessionist conflict}. \textit{Journal of Peace Research} 54(3): 365--381.
	% \item Katja B. Kleinberg and Benjamin O. Fordham (2010) \href{https://doi.org/10.1177/0022002710364128}{Trade and foreign policy attitudes}. \textit{Journal of Conflict Resolution} 54(5): 687--714.
  % \item Rodrik, Dani (2021) \href{https://doi.org/10.1146/annurev-economics-070220-032416}{Why Does Globalization Fuel Populism? Economics, Culture, and the Rise of Right-Wing Populism}. \textit{Annual Review of Economics} 13: 133--170.
\end{itemize}

% • Oneal, John R., and Bruce Russett. “Assessing the liberal peace with alter-
% native specifications: Trade still reduces conflict.” Journal of Peace Research
% 36.4 (1999): 423-442.
% • Barbieri, Katherine, and Jack S. Levy. “Sleeping with the enemy: The impact
% of war on trade.” Journal of Peace Research 36.4 (1999): 463-479.
% • Nielsen, Richard A., et al. “Foreign aid shocks as a cause of violent armed
% conflict.” American Journal of Political Science 55.2 (2011): 219-232.


\vspace{20pt}
\hrule
% ------------------------------------------------------------
\subsection*{Session 7: Disaggregating and predicting conflict}

\noindent\textit{Required:}

\begin{itemize}
	\item Güne\c{s} Murat Tezcür (2016) \href{https://doi.org/10.1017/S0003055416000150}{Ordinary People, Extraordinary Risks: Participation in an Ethnic Rebellion}. \textit{American Political Science Review} 110(2): 247--264.
	\item Nils W. Metternich, Cassy Dorff, Max Gallop, Simon Weschle, and Michael D. Ward (2013) \href{https://doi.org/10.1111/ajps.12039}{Antigovernment Networks in Civil Conflicts: How Network Structures Affect Conflictual Behavior}. \textit{American Journal of Political Science} 57(4): 892--911.
	\item Jason Lyall (2009) \href{https://doi.org/10.1177/0022002708330881}{Does Indiscriminate Violence Incite Insurgent Attacks? Evidence from Chechnya}. \textit{Journal of Conflict Resolution} 53(3): 331--362.
\end{itemize}

% Mampilly, Z., & Stewart, M. A. (2021). A Typology of Rebel Political Institutional  Arrangements. Journal of Conflict Resolution, 65(1), 15–45. https://doi.org/10.1177/0022002720935642

% Staniland (2012) Organizing Insurgency: Networks, Resources, and Rebellion in South Asia

\noindent\textit{Complementary:}

\begin{itemize}
	\item Macartan Humphreys and Jeremy M. Weinstein (2008) \href{https://doi.org/10.1111/j.1540-5907.2008.00322.x}{Who Fights? The Determinants of Participation in Civil War}. \textit{American Journal of Political Science} 52(2): 436--455.
	\item Matthew Adam Kocher, Thomas B. Pepinsky and Stathis N. Kalyvas (2011) \href{https://doi.org/10.1111/j.1540-5907.2010.00498.x}{Aerial Bombing and Counterinsurgency in the Vietnam War}. \textit{American Journal of Political Science} 55(2): 201--218.
  % \item Lars-Erik Cederman, Simon Hug, Livia I. Schubiger, and Francisco Villamil (2020) \href{https://doi.org/10.1177/0022002719898873}{Civilian Victimization and Ethnic Civil War}. \textit{Journal of Conflict Resolution} 64(7–8): 1199--1225.
	\item Monica D. Toft and Yuri M. Zhukov (2015) \href{https://doi.org/10.1017/S000305541500012X}{Islamists and Nationalists: Rebel Motivation and Counterinsurgency in Russia's North Caucasus}. \textit{American Political Science Review} 109(2): 222--238.
  \item Samuel Bazzi, Robert A. Blair, Christopher Blattman, Oeindrila Dube, Matthew Gudgeon, and Richard Peck (2022) \href{https://doi.org/10.1162/rest_a_01016}{The Promise and Pitfalls of Conflict Prediction: Evidence from Colombia and Indonesia}. \textit{The Review of Economics and Statistics} 104(4): 764--779.
\end{itemize}

% Ward et al (2013) Learning from the Past and Stepping into the Future: Toward a New Generation of Conflict Prediction


\vspace{20pt}
\hrule
% ------------------------------------------------------------
\subsection*{Session 8: Wartime violence}

\noindent\textit{Required:}

\begin{itemize}
  \item Stathis Kalyvas (2006) \href{https://bibliotecas.uc3m.es/permalink/f/1nggclj/34UC3M_ALMA21161986050004213}{\textit{The Logic of Violence in Civil War}}. Cambridge University Press, 2006. Chapters 1 (16--31) and 7 (173--209).
  \item Laia Balcells (2010) \href{https://doi.org/10.1111/j.1468-2478.2010.00588.x}{Rivalry and Revenge: Violence against Civilians in Conventional Civil Wars}. \textit{International Studies Quarterly} 54(2): 291--313.
  \item Nils B. Weidmann (2011) \href{https://doi.org/10.1017/S0022381611000831}{Violence ``from above'' or ``from below''? The Role of Ethnicity in Bosnia’s Civil War}. \textit{The Journal of Politics} 73(4): 1178--1190.
\end{itemize}

\noindent\textit{Complementary:}

\begin{itemize}
	\item Dara K. Cohen (2013) \href{https://doi.org/10.1017/S0003055413000221}{Explaining Rape during Civil War: Cross-National Evidence (1980–2009)}. \textit{American Political Science Review} 107(3): 461--477.
  \item Robert Braun (2016) \href{https://doi.org/10.1017/S0003055415000544}{Religious Minorities and Resistance to Genocide: The Collective Rescue of Jews in the Netherlands during the Holocaust.} \textit{American Political Science Review} 110(1): 127--147.
  \item Stathis Kalyvas and Matthew A. Kocher (2007) \href{https://doi.org/10.1353/wp.2007.0023}{How `Free' is Free Riding in Civil Wars?: Violence, Insurgency, and the Collective Action Problem}. \textit{World Politics} 59(2): 177--216.
\end{itemize}

\vspace{20pt}
\hrule
% ------------------------------------------------------------
\subsection*{Session 9: Terrorism}

\noindent\textit{Required:}


\begin{itemize}
  % \item Luis de la Calle and Ignacio Sánchez-Cuenca (2021) \textit{Underground violence: A theory of terrorism.} Unpublished book manuscript. Chapters Introduction, 1 and 2 (pp. 2--75).
	\item David Lake (2002) \href{https://doi.org/10.1017/S777777770200002X}{Rational Extremism: Understanding Terrorism in the Twenty First Century}. \textit{International Organization} 56(1): 15--29.
  \item Luis de la Calle and Ignacio Sánchez-Cuenca (2011) \href{https://doi.org/10.1177/0032329211415506}{What We Talk About When We Talk About Terrorism}. \textit{Politics \& Society} 39(3): 451--472.
  \item Robert Pape (2003) \href{https://doi.org/10.1017/S000305540300073X}{The Strategic Logic of Suicide Terrorism}. \textit{American Political Science Review} 97(3): 343--361.
\end{itemize}

\noindent\textit{Complementary:}

% \item Virginia P. Fortna (2015) Do Terrorists Win? Rebels’ Use of Terrorism and Civil War Outcomes. \textit{International Organization} 69(3): 519-556.

\begin{itemize}
	\item Andrew. H Kydd and Barbara F. Walter (2006) \href{https://doi.org/10.1162/isec.2006.31.1.49}{The Strategies of Terrorism}. \textit{International Security} 31(1): 49--80.
	\item Burcu Savun and Brian J. Phillips (2009) \href{https://doi.org/10.1177/0022002709342978}{Democracy, Foreign Policy, and Terrorism}. \textit{Journal of Conflict Resolution} 53(6): 878--904.
	\item Barbara F. Walter (2017) \href{https://doi.org/10.1162/ISEC_a_00292}{The Extremist’s Advantage in Civil Wars}. \textit{International Security} 42(2): 7--39.
  \item Ana Arjona, Nelson Kasfir and Zachariah Mampilly (2015) ``Introduction.'' In: \href{https://www.cambridge.org/core/books/rebel-governance-in-civil-war/C40247AED4FA30DC2704EB64EA5CFFD5}{\textit{Rebel governance in civil war}}. New York: Cambridge University Press. Chapter 1 (pp. 1-20).
\end{itemize}

\vspace{20pt}
\hrule
% ------------------------------------------------------------
\subsection*{Session 10: Legacies of conflict}

\noindent\textit{Required:}

\begin{itemize}
  \item Christopher Blattman (2009) \href{https://doi.org/10.1017/S0003055409090212}{From Violence to Voting: War and Political Participation in Uganda}. \textit{American Political Science Review} 103(2): 231--247.
  \item Noam Lupu and Leonid Peisakhin (2017) \href{https://doi.org/10.1111/ajps.12327}{The Legacy of Political Violence across Generations}. \textit{American Journal of Political Science} 61(4): 836--851.
  \item Arturas Rozenas and Yuri M. Zhukov (2019) \href{https://doi.org/10.1017/S0003055419000066}{Mass Repression and Political Loyalty: Evidence from Stalin’s ‘Terror by Hunger’}. \textit{American Political Science Review} 113(2): 569--583.
\end{itemize}

\noindent\textit{Complementary:}

\begin{itemize}
	\item Volha Charnysh and Leonid Peisakhin (2021) \href{https://doi.org/10.1017/S0007123420000447}{The Role of Communities in the Transmission of Political Values: Evidence from Forced Population Transfers}. \textit{British Journal of Political Science} 52(1): 238--258.
  \item Summer Lindsey (2022) \href{https://doi.org/10.1111/ajps.12637}{Conflict, Protection, and Punishment: Repercussions of Violence in Eastern DR Congo}. \textit{American Journal of Political Science} 66(1): 187--204.
	\item Robert A. Blair, Manuel Moscoso-Rojas, Andrés Vargas Castillo, and Michael Weintraub (2022) \href{https://doi.org/10.1017/S0003055422000284}{Preventing Rebel Resurgence after Civil War: A Field Experiment in Security and Justice Provision in Rural Colombia}. \textit{American Political Science Review} 116(4): 1258--1277.
\end{itemize}

\vspace{20pt}
\hrule
% ------------------------------------------------------------
\subsection*{Session 11: Constructivism, norm diffusion, and TJ}

\noindent\textit{Required:}

\begin{itemize}
	\item Martha Finnemore and Kathryn Sikkink (1998) \href{https://doi.org/10.1162/002081898550789}{International Norm Dynamics and Political Change}. \textit{International Organization} 52(4): 887--917.
  \item Hunjoon Kim (2012) \href{https://doi.org/10.1177/0022343311431600}{Structural determinants of human rights prosecutions after democratic transition}. \textit{Journal of Peace Research} 49(2): 305---320.
	\item Hunjoon Kim and Kathryn Sikkink (2010) \href{https://doi.org/10.1111/j.1468-2478.2010.00621.x}{Explaining the Deterrence Effect of Human Rights Prosecutions for Transitional Countries}. \textit{International Studies Quarterly} 54(4): 939--963.
\end{itemize}

% Wendt, Alexander. 1992. Anarchy Is What States Make Of It: The Social Construction of Power Politics. International Organization 46 (4): 887-918.

\clearpage
\noindent\textit{Complementary:}

\begin{itemize}
	% \item Emmanuel Adler (1997) \href{https://doi.org/10.1177/1354066197003003003}{Seizing the Middle Ground: Constructivism in World Politics}. \textit{European Journal of International Relations} 3(3): 319--363.
	\item James Lebovic and Erik Voeten (2006) \href{https://doi.org/10.1111/j.1468-2478.2006.00429.x}{The Politics of Shame: The Condemnation of Country Human Rights Practices in the UNHCR}. \textit{International Studies Quarterly} 50(4): 861--888.
  \item Kathryn Sikkink and Carrie Booth Walling (2007) \href{https://doi.org/10.1177/0022343307078953}{The Impact of Human Rights Trials in Latin America.} \textit{Journal of Peace Research} 44(4): 427--445.
	\item Tricia D. Olsen, Leigh A. Payne and Andrew G. Reiter (2010) \href{https://doi.org/10.1353/hrq.2010.0021}{The Justice Balance: When Transitional Justice Improves Human Rights and Democracy}. \textit{Human Rights Quarterly} 32(4): 980--1007.
	% \item Peter J. Katzenstein (2003) \href{https://doi.org/10.1017/S0020818303574033}{Same War-Different Views: Germany, Japan, and Counterterrorism}. \textit{International Organization} 57(4): 731--760.
\end{itemize}

\vspace{20pt}
\hrule
% ------------------------------------------------------------
\subsection*{Session 12: Research workshop}

No readings. Presentation of research projects and workshop discussion.

\vspace{20pt}
\hrule
% ------------------------------------------------------------


% \clearpage
% \subsection*{Extra (depending on schedule): International borders}

% \noindent\textit{Required/complementary:}

% \begin{itemize}
% 	\item David B. Carter and Hein E. Goemans (2011) \href{https://doi.org/10.1017/S0020818311000051}{The Making of the Territorial Order: New Borders and the Emergence of Interstate Conflict}. \textit{International Organization} 65(2): 275--309.
% 	\item Scott F. Abramson (2017) \href{https://doi.org/10.1017/S0020818316000308}{The Economic Origins of the Territorial State}. \textit{International Organization} 71(1): 97--130.
% 	\item Michael Mann (2004) \href{https://www.cambridge.org/core/books/dark-side-of-democracy/7E75A132A188A2804E91F4F209B6FE1F}{\textit{The Dark Side of Democracy: Explaining Ethnic Cleansing.}} New York: Cambridge University Press. [Chapter 1, pp. 1--33.]
% 	\item Carl Müller-Crepon, Guy Schvitz, and Lars-Erik Cederman (2021) \href{http://www.carlmueller-crepon.org/publication/state_shape/}{Shaping States into Nations: The Effects of Ethnic Geography on State Borders}. Unpublished manuscript.
% 	\item Andreas Wimmer (2008) \href{https://doi.org/10.1086/522803}{The Making and Unmaking of Ethnic Boundaries: A Multilevel Process Theory}. \textit{American Journal of Sociology} 113(4): 970--1022.
% \end{itemize}

% \noindent\textit{Complementary:}
%
% \begin{itemize}
%   \item
% \end{itemize}

\end{document}
