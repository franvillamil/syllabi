\documentclass[12pt, a4paper]{article}
\usepackage[margin = 2cm]{geometry}
\usepackage{graphicx, booktabs}
\usepackage[english]{babel}
\usepackage[utf8]{inputenc}
\usepackage{color}
\definecolor{dark_blue}{rgb}{0.2, 0.0, .7}
\usepackage[colorlinks = TRUE, linkcolor = dark_blue]{hyperref}
\usepackage{setspace}
\setstretch{1.25}
\renewcommand*\rmdefault{ppl}

\usepackage[]{titlesec}
    \titleformat*{\section}{\large\bf}
    \titleformat*{\subsection}{\normalsize\bf}
    \titleformat*{\subsubsection}{\normalsize\it}

%%%%%%%%%%%%%%%%%%%%%%%%%%%%%%%%%%%%
\begin{document}
\begin{center}
{\LARGE\bfseries Política Mundial}\\\vspace{10pt}
Grupos 88 y 188 (CCPP y Sociología)\\
Universidad Carlos III de Madrid\\\vspace{10pt}
{\large\bfseries Segundo cuatrimestre 2025--2026}\\\vspace{15pt}
Web: \href{https://franvillamil.github.io/pol_mundial/}{franvillamil.github.io/pol\_mundial/}
\end{center}

\section{Información y horarios}

\begin{minipage}{0.39\textwidth}
\flushleft
\textbf{Profesor:}\\
Francisco Villamil\\
\href{francisco.villamil@uc3m.es}{francisco.villamil@uc3m.es}\\
Despacho: 18.2.A34\\
Tutorías: cita por email
\end{minipage}\hfill
\begin{minipage}{0.39\textwidth}
\flushleft
\textbf{Profesor seminarios:}\\
Luis Azores\\
\href{lazores@pa.uc3m.es}{lazores@pa.uc3m.es}\\
Despacho:\\
Tutorías: cita por email
\end{minipage}

\vspace{15pt}

\begin{center}
\begin{tabular}{lll}
& \textbf{Magistral} & \textbf{Seminario} \\
\hline
Grupo 88 & Martes, 9:00-10:30 & Viernes 10:45-12:15 \\
  & Aula 9.2.03 & Aula 15.0.16 \\
  \hline
Grupo 188 & Martes, 16:15-17:45 & Martes 18:00-19.30 \\
  & Aula 4.1.03 & Aula 4.1.06 \\
  \hline
\end{tabular}
\end{center}


\vspace{10pt}
\section{Description}



\section{Requisitos generales y recursos}

\begin{itemize}
    \item[] \textbf{Libro de texto principal:}
    \item Jeffry A. Frieden, David A. Lake, and Kenneth A. Schultz, \textit{World Politics: Interests, Interactions, Institutions.} Ed: W.W. Norton (actualmente usando la 4ta edición, 2019).
    \item[] \textbf{Lecturas complementarias:}
    \item Se irán añadiendo a la página web (\href{https://franvillamil.github.io/pol_mundial/resources.html}{franvillamil.github.io/pol\_mundial/resources.html}) así como a Aula Global, tanto las de apoyo a las clases magistrales como las que se utilicen para los seminarios.
\end{itemize}

% \newpage
\section{Evaluación}

\begin{itemize}
\setlength\itemsep{0pt}
  \item \textbf{20\%}: Actividades seminarios
  \item \textbf{30\%}: Presentación
  \item \textbf{50\%}: Examen final
\end{itemize}

\subsection*{Actividades seminarios (30\%)}

\subsection*{Presentación (20\%)}

\subsection*{Examen final (50\%)}

El examen consistirá en dos preguntas a desarrollar que tendrán que escribirse a mano en clase. El examen tendrá una duración de 1 hora. Cada una de las preguntas tendrá un límite corto de extensión definido, para priorizar que se planteen de antemano como un argumento corto y razonado.

Se podrán utilizar libros, apuntes o documentos durante su realización, siempre y cuando estos estén en papel. \textbf{No se permitirá el uso de dispositivos electrónicos} durante el examen, y cualquier detección de plagio o uso de AI conlleva un suspenso. %Al final del examen, se elegirá aleatoriamente a dos estudiantes para que explican oralmente sus respuestas.

\subsection*{Examen parcial}


\section{Programa}

Resumen y calendario disponible en la web del curso: \href{https://franvillamil.github.io/pol_mundial/}{franvillamil.github.io/pol\_mundial/}

\subsection*{1. Introducción [PDF] No seminario}
\subsection*{2. El mundo hoy   Lectura/Datos?}
\subsection*{3. Actores e intereses    Actividad?}
\subsection*{4. El problema de la guerra   Lectura}
\subsection*{5. Guerras y organizaciones   Lectura}
\subsection*{6. Terrorismo   Lectura}
\subsection*{7. Leyes, normas y DDHH   Lectura}
\subsection*{8. Examen parcial   Datos overview}
\subsection*{9. Comercio internacional   Lectura}
\subsection*{10. Finanzas y dinero   Datos?}
\subsection*{11. Desarrollo económico    Lectura?}
\subsection*{12. Futuro y problemas emergentes   Presentaciones}
\subsection*{13. Resumen/Presentaciones    Presentaciones}


\end{document}
