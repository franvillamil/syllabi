





% NOTE por 2023: igual mejor hacer presentations + essay by groups, trabajar en grupos DESDE EL PRINCIPIO (y tbm los ejercicios?)

\documentclass[12pt, a4paper]{article}
\usepackage[margin = 2cm]{geometry}
\usepackage{graphicx}
\usepackage[english]{babel}
\usepackage[utf8]{inputenc}
\usepackage[colorlinks = TRUE]{hyperref}
\usepackage{setspace}
\setstretch{1.25}
\renewcommand*\rmdefault{ppl}


\usepackage[]{titlesec}
    \titleformat*{\section}{\large\bf}
    \titleformat*{\subsection}{\normalsize\bf}
    \titleformat*{\subsubsection}{\normalsize\it}

%%%%%%%%%%%%%%%%%%%%%%%%%%%%%%%%%%%%
\begin{document}
\begin{center}
{\LARGE\bf Research Design}\\\vspace{10pt}
Master in Computational Social Science\\
Universidad Carlos III de Madrid\\\vspace{10pt}
{\large Fall 2023}\\
\end{center}

\vspace{20pt}

\begin{minipage}{0.49\textwidth}
\centering
Francisco Villamil\\
\href{francisco.villamil@uc3m.es}{francisco.villamil@uc3m.es}\\
Office: 18.2.D15 (Getafe)\\
Office hours by appointment
\end{minipage}\hfill
\begin{minipage}{0.49\textwidth}
\centering
\textbf{Tuesdays 18h-21h/15h-18h}\\Room 2.A.04 (Puerta de Toledo)\\September 12th - October 24th\\
\end{minipage}


\vspace{10pt}
\section{Description}

This course provides an introduction to research design in the social sciences.
The goal is to equip students with the skills necessary to evaluate and develop strategies to answer empirical questions with data.
We will not cover statistical techniques or advanced data analysis.
Rather, our focus will be on the logic of empirical comparison, particularly applied to quantitative data.

\section{Requirements}

We meet once a week for a total of seven sessions. During the first four weeks, we will review the main topics in research design. Every session will include a lecture followed by a discussion of a research paper. During sessions 5--6 we will have a workshop where students both present in group an ongoing project and comment on each other's projects. Attendance to all sessions is mandatory. During last session we will cover more advance topics, give an overview of the course, and reserve time for questions or to review some topics.

\section{Materials}

This course does not follow any textbook in particular.
However, there are some books that were used to design the course and can be very useful to expand on what is covered in class (books by Huntington-Klein and Cunningham are available online for free):

\begin{itemize}
\setlength\itemsep{-5pt}
  \item Nick Huntington-Klein, \href{https://theeffectbook.net/}{\textit{The Effect: An Introduction to Research Design and Causality}} (Chapman and Hall/CRC, 2021).
  \item Kosuke Imai, \textit{Quantiative Social Science: An Introduction} (Princeton UP, 2017).
  \item Dimiter Toshkov, \textit{Research Design in Political Science} (Palgrave, 2016)
  \item Scott Cunningham, \href{https://mixtape.scunning.com/}{\textit{Causal Inference: The Mixtape}} (Yale University Press, 2021).
\end{itemize}

% Some of these books (Huntington-Klein's \textit{The Effect} and Cunningham's \textit{Mixtape}) are available online for free.\footnote{\url{https://theeffectbook.net/} and \url{https://mixtape.scunning.com/}.} Besides, I will make available additional readings in \textit{Aula Global}.

\section{Assessment}

% Students will be evaluated based on three different assignments:

\subsection*{Participation (5\%)}

Every student is expected to be an active participant in all sessions, asking questions and engaging in discussions, including during the lectures.

\subsection*{Research papers reviews (15\%)}

In sessions 2--4, we will discuss a research paper in the second half of class. Students are expected to bring in or send before class a short commentary (no summary, just a few discussion points). Each one is 5\%.

\subsection*{Workshop presentation (20\%)}

Sessions 5 and 6 will be a workshop where students present in group an ongoing project, corresponding to their final essay. This project should present a research question that can be answered empirically with quantitative data and a strategy to answer it. Rather than focusing on data analysis, the focus should be on the type of variation that will be exploited and how well it answers the question. All the aspects that we covered in class should be discussed. A potential structure could be:

\begin{itemize}
\setlength\itemsep{-5pt}
  \item Present a topic in general terms, why is it relevant?
  \item Specific research question
  \item Empirical strategy:
  \vspace{-10pt}
  \begin{itemize}
  \setlength\itemsep{-5pt}
    \item What type of variation is going to be exploited? Unit of analyses
    \item Data: source, variables, measurement
    \item Empirical comparisons
  \end{itemize}
  \item External validity: how do the findings travel to other contexts?
  \item How much more of the original topic we now know?
\end{itemize}

\subsection*{Workshop feedback (10\%)}

Student will also comment on each other's project, looking for limitations and possible ways forward.

\subsection*{Final essay (40\%)}

The main assignment is a written essay developing the research design. This assignment can be thought of as a pre-analysis plan for some study, but there are other options (e.g. compare two different strategies for the same research question).

\newpage
\section{Course outline}

% \hline % ============================================================
\subsection*{Session 1: Introduction to research design}

\begin{itemize}
\setlength\itemsep{-5pt}
  \item[-] What is a research design and why do we need it?
  \item[-] Different types of research
  \item[-] Research Questions
  \item[-] Role of theory
\end{itemize}

\textit{Reading:}

\begin{itemize}
  \item Jill Lepore. \href{https://www.newyorker.com/magazine/2023/04/03/the-data-delusion}{The Data Delusion}. \textit{The New Yorker}, 03/04/2023.
\end{itemize}


% \hline % ============================================================
\subsection*{Session 2: Basics of quantitative research}

\begin{itemize}
\setlength\itemsep{-5pt}
  \item[-] Qualitative and quantitative data
  \item[-] Empirical relationships
  \item[-] Description and explanation
  \item[-] Concepts, operationalization, and measurement
\end{itemize}

\textit{Reading:}

\begin{itemize}
  \item Carl Müller-Crepon, Philipp Hunziker, and Lars-Erik Cederman. \href{https://journals.sagepub.com/doi/10.1177/0022002720963674}{Roads to Rule, Roads to Rebel: Relational State Capacity and Conflict in Africa.} \textit{Journal of Conflict Resolution} 65(2--3): 563--590.
\end{itemize}

% \hline % ============================================================
\subsection*{Session 3: Understanding causality}


% \hline % ============================================================
\subsection*{Session 4: Causal identification with observational data}


% \hline % ============================================================
\subsection*{Session 5 \& 6: Workshop}


% \hline % ============================================================
\subsection*{Session 7: Advanced methods, overview, questions}




% % \hline % ============================================================
% \subsection*{September 13th: Introduction to social research}
%
% \begin{itemize}
% \setlength\itemsep{-5pt}
%   \item What is social \textit{science}?
%   \item Using empirical evidence to answer questions
%   \item Importance of research design
%   \item Types of research questions and types of empirical research
% \end{itemize}
%
% \subsection*{September 20th: Social mechanisms and causality}
%
% \begin{itemize}
% \setlength\itemsep{-5pt}
%   \item What are we explaining? Building blocks: units, variables, processes
%   \item Simplifying social complexity
%   \item Different units of analysis and mechanisms
%   \item How to approach this empirically? Key idea: variability
% \end{itemize}
%
% \subsection*{September 27th: Understanding empirical evidence}
%
% \begin{itemize}
% \setlength\itemsep{-5pt}
%   \item What is \textit{data}? Types of empirical evidence
%   \item Unit of analysis and variability, different approaches %Case studies, comparative studies, quantitative studies with large samples
%   \item Raw data and final data, aggregating and disaggregating, measurement problems
% \end{itemize}
%
% \subsection*{October 4th: Causal inference design}
%
% \begin{itemize}
% \setlength\itemsep{-5pt}
%   \item The experimental ideal in the natural and social sciences
%   \item How to get closer with observational data?
%   \item Confounding, selection bias, collider bias, etc
% \end{itemize}
%
% \subsection*{October 11th: Assessing external and internal validity}
%
% \begin{itemize}
% \setlength\itemsep{-5pt}
%   \item Bringing it all together
%   \item Where does evidence come from? External validity
%   \item Issues with unit of analysis? Ecological fallacy
%   \item Unaccounted for processes? Diffusion, unit independence
% \end{itemize}
%
% \subsection*{October 18th: Introduction to causal inference methods}
%
% \begin{itemize}
% \setlength\itemsep{-5pt}
%   \item Understanding the logic of methods to determine causality, without the use of statistics
%   \item Controls, matching
%   \item Exploiting exogeneity: difference-in-differences, regression discontuinity
% \end{itemize}
%
% \noindent
% \textbf{(*)} Depending on timing, there will be a final session on \textbf{quantitative research workflow}.
%
% \subsection*{October 25th: Presentations workshop}

\subsection*{$*$ Final essay {\color{red}{deadline}}: November 3rd, 23.59h}

\end{document}
