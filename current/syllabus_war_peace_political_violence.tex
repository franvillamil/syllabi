%%% AY 2023-24:
% - Formar grupos en la primera semana (calcular numero)
% - Hacer una presentación cada seminario + debate posterior
% - Y entregar 2 response papers individuales en días distintos al de la presentación
% - Dejar un seminario al final para recuperar?

\documentclass[12pt, a4paper]{article}
\usepackage[margin = 2cm]{geometry}
\usepackage{graphicx, booktabs}
\usepackage[english]{babel}
\usepackage[utf8]{inputenc}
\usepackage{color}
\definecolor{dark_blue}{rgb}{0.2, 0.0, .7}
\usepackage[colorlinks = TRUE, linkcolor = dark_blue]{hyperref}
\usepackage{setspace}
\setstretch{1.25}
\renewcommand*\rmdefault{ppl}


\usepackage[]{titlesec}
    \titleformat*{\section}{\large\bf}
    \titleformat*{\subsection}{\normalsize\bf}
    \titleformat*{\subsubsection}{\normalsize\it}

%%%%%%%%%%%%%%%%%%%%%%%%%%%%%%%%%%%%
\begin{document}
\begin{center}
{\LARGE\bfseries War, peace, and political violence}\\\vspace{10pt}
BA History and Politics\\
Universidad Carlos III de Madrid\\\vspace{10pt}
{\large\bfseries Fall 2023}\\
\end{center}

\vspace{20pt}

\begin{minipage}{0.54\textwidth}
\flushleft
Francisco Villamil\\
\href{francisco.villamil@uc3m.es}{francisco.villamil@uc3m.es}\\
Office: 18.2.D15\\
Office hours: by appointment
\end{minipage}\hfill
\begin{minipage}{0.39\textwidth}
\flushleft
\textbf{Lecture}: Thursday 14h30--16h\\
\textbf{Seminar}: Friday 14h30--16h\\\vspace{5pt}
Room 14.1.02\\
\href{https://franvillamil.github.io/wp_polvio/}{franvillamil.github.io/wp\_polvio/}
\end{minipage}


\vspace{10pt}
\section{Description}

This course covers a wide range of topics in conflict research, including inter-state wars, civil wars, and causes and dynamics of political violence.
Its main goal is to provide students with the conceptual and theoretical tools to think analytically about conflicts and political violence. Some of the questions we will explore are: Why do countries fight each other?
% How do changes in the international system impact conflicts across the world?
What explains the outbreak of civil wars? Why and how civilians are killed during wars? What are the long-term consequences of conflicts?

\section{General requirements and materials}

We meet twice a week. In the lectures, we will review the main debates in each topic. Each lecture has one reading assigned, usually a research article, that covers part or most of what we will talk about. Reading it is not mandatory, but recommended, either before or after the lecture. In each seminar, we will discuss a reading related to the lecture. These readings, which are \textbf{mandatory}, are shorter and lighter than the ones assigned to the lectures, and are meant to reflect or expand on the topic covered each week. In addition, we will have a group presentation in most seminar days that discusses something related to the reading or week topic.
Slides and general information can be found at \href{https://franvillamil.github.io/wp_polvio/}{https://franvillamil.github.io/wp\_polvio/}. All readings will be uploaded to \href{https://aulaglobal.uc3m.es/}{\textit{Aula Global}}.

% \newpage
\section{Assessment and grading}

\begin{itemize}
\setlength\itemsep{0pt}
  \item \textbf{10\%}: Participation
  \item \textbf{15\%}: Reading memos
  \item \textbf{15\%}: Group presentation
  \item \textbf{60\%}: Final exam
\end{itemize}

\subsection*{Participation (10\%)}

Everyone is expected to attend all sessions and be an active participant in the discussions, especially in the seminar sessions.

\subsection*{Reading memos (15\%)}

Students have to send \textbf{up to three reading memos} of the seminar readings, excluding the week they choose to do the group presentation. Reading memos are just a half-page text with ideas from the text to discuss in class (no need to write a summary). They must be sent by email \textbf{before class starts}.

\subsection*{Presentation (15\%)}

In groups of 3-4 people, students must give a 15-min group presentation during the seminar day on something related to the week topic. It can be directly about the seminar reading or on something related (e.g. the case or the question).

Topics should be briefly discussed beforehand and dates will have handled on a first come, first served basis. There is the option of doing a two presentations during the last seminar day of the course for those who did not choose in time, but it has a 25\% grade penalty.

\subsection*{Final exam (60\%) -- Two options:}

\begin{itemize}
  \item[\textbf{A.}] A final take-home exam. Questions will be handed out at least 24h before deadline. Its goal is to evaluate how well students understood the main concepts and ideas.
  \item[\textbf{B.}] A book review of a relevant book, commenting some of the topics discussed in class \textbf{(max 2,500 words)}. Some of the pre-approved options are:
  {\small
  \item[-] P. Radden Keefe, \textit{Say Nothing: A True Story of Murder and Memory in Northern Ireland} \vspace{-10pt}
  \item[-] A. Gopal, \textit{No Good Men Among the Living: America, the Taliban, and the War Through Afghan Eyes} \vspace{-10pt}
  \item[-] W. Finnegan, \textit{A Complicated War: The Harrowing of Mozambique} \vspace{-10pt}
  \item[-] S. Subramanian, \textit{This Divided Island: Life, Death, and the Sri Lankan War} \vspace{-10pt}
  \item[-] V. Bevins, \textit{The Jakarta Method: Washington's Anticommunist Crusade and the Mass Murder Program that Shaped Our World} \vspace{-10pt}
  \item[-] J. Warrick, \textit{Black Flags: The Rise of ISIS}
  }
  \item[] (Any other option is possible, but needs to be \textbf{previously approved})
\end{itemize}

\newpage
\section{Course outline}

\begin{table*}[!ht]
  \centering
  \setstretch{1.4}

  \begin{tabular}{lll}
  % \hline
    \textbf{Week 1} & Sept 7-8   & \hyperref[intro]{Introduction} \\
    \textbf{Week 2} & Sept 14-15 & \hyperref[ir_basics]{Basics of IR} \\
    \textbf{Week 3} & Sept 21-22 & \hyperref[interstate]{Understanding interstate war} \\
    \textbf{Week 4} & Sept 28-29 & \hyperref[war_history]{War throughout history} \\
    \textbf{Week 5} & Oct 5-6    & \hyperref[cw1]{Civil wars I} \\
    \textit{No class} & Oct 12-13  &\\
    \textbf{Week 6} & Oct 19-20  & \hyperref[cw2]{Civil wars II} \\
    \textbf{Week 7} & Oct 26-27  & \hyperref[wartime_violence]{Wartime violence} \\
    \textbf{Week 8} & Nov 2-3    & \hyperref[rebels]{Non-state armed actors and civilians} \\
    \textbf{Week 9} & Nov 9-10   & \hyperref[rebels]{Terrorism} \\
    \textbf{Week 10} & Nov 16-17  & \hyperref[terrorism]{Postwar politics and prevention}  \\
    \textbf{Week 11} & Nov 23-24  & \hyperref[postwar]{Transitional Justice} \\
    \textbf{Week 12} & Nov 30-31  & \hyperref[tj]{Legacies and consequences of war} \\
    \textit{No class} & Dec 7-8    &\\
    \textbf{Week 13} & Dec 14-15  & \hyperref[wrap_up]{Overview, wrap-up, extra presentations, etc} \\
  % \hline
  \end{tabular}
\end{table*}



\hline % ============================================================

\subsection{Introduction} \label{intro}

Presentation. Course structure and organizational issues. Introduction: what is political violence and what are we going to talk about? \textit{No readings.}\\

\hline % ============================================================

\subsection{Basics of IR}\label{ir_basics}

The three visions in IR, understanding cooperation and conflict at the international level.

\subsubsection*{Lecture}

\begin{itemize}
\setlength\itemsep{0pt}
\item Stephen M. Walt, `\href{https://doi.org/10.2307/1149275}{International Relations: One World, Many Theories.}' \textit{Foreign Policy} 110: 29--32, 34--46, 1998.
\end{itemize}

\subsubsection*{Seminar}

\begin{itemize}
  \item Stathis Kalyvas, `\href{https://iai.tv/articles/how-we-got-putin-so-wrong-auid-2063}{How we got Putin so wrong}.` \textit{iai news}, 01/03/2022.
  \item Max Abrahms, `\href{https://www.theatlantic.com/ideas/archive/2023/03/russia-ukraine-war-pundits-history-international-relations/673293/}{I Teach International Relations. I Think We’re Making a Mistake in Ukraine.}' \textit{The Atlantic}, 07/03/2023.
\end{itemize}

\hline % ============================================================

\subsection{Understanding interstate war}\label{interstate}

Why do countries go to war? Different explanations at different levels: issues, ecological factors, rationalist approaches. Democratic and capitalist peace.

\subsubsection*{Lecture}

\begin{itemize}
\setlength\itemsep{0pt}
% \item Chapter 1 in Kalevi J Holsti, \href{https://doi.org/10.1017/CBO9780511628290}{\textit{Peace and War: Armed Conflicts and International Order, 1648-1989.}} Cambridge UP, 1991: pp. 1--24.
\item Jack S. Levy, `Interstate war and peace.' In \href{https://uk.sagepub.com/en-gb/eur/handbook-of-international-relations/book234093}{Handbook of International Relations}, 2013, pp. 581--606.
\end{itemize}

\subsubsection*{Seminar}

\begin{itemize}
\setlength\itemsep{0pt}
\item Dexter Filkins, `\href{https://www.newyorker.com/magazine/2022/11/21/a-dangerous-game-over-taiwan}{A dangerous game over Taiwan}.' \textit{The New Yorker,} 21/11/2022.
\end{itemize}

\hline % ============================================================

\subsection{War throughout history}\label{war_history}

How war has changed throughout history due to two main phenomena: the emergence of the modern state and the nation-state. State-building and war: origins of the state, role of international conflict in the creation of states. The development of nationalisms and its relationship with political violence.

\subsubsection*{Lecture}

\begin{itemize}
\setlength\itemsep{0pt}
\item Chapter 1 in Charles Tilly, \href{https://www.wiley.com/en-us/Coercion%2C+Capital+and+European+States%2C+A+D+990+1992-p-9781557863683}{\textit{Coercion, capital, and European states, AD 990-1992}}. Wiley-Blackwell, 1993: pp. 1--37.
\item Andreas Wimmer and Brian Min, `\href{https://doi.org/10.1177/000312240607100601}{From empire to nation-state: Explaining wars in the modern world, 1816-2001}.' \textit{American Sociological Review} 71: 867--897, 2006.
\end{itemize}

\subsubsection*{Seminar}

\begin{itemize}
\setlength\itemsep{0pt}
\item John Reed, Guy Chazan \& Roman Olearchyk, `\href{https://www.ft.com/content/9ab50dee-67f5-4e1b-8456-d8f11814ef18}{The birth of a new Ukraine: how Russia's war united a nation}' \textit{Finantial Times,} 17/03/2022.
\end{itemize}


\hline % ============================================================

\subsection{Civil wars I}\label{cw1}

Basic concepts and types of civil wars. After 1990, there is a deep increase in the outbreak of civil wars. What used to be explained as popular revolutions, now is seen as a problem of anarchy, looting, and greed.

\subsubsection*{Lecture}

\begin{itemize}
\setlength\itemsep{0pt}
\item James Fearon \& David Laitin, `\href{https://doi.org/10.1017/S0003055403000534}{Ethnicity, insurgency, and civil war}.' \textit{American Political Science Review} 97(1): 75--90, 2003.
\end{itemize}

\subsubsection*{Seminar}

\begin{itemize}
\setlength\itemsep{-5pt}
\item Robert D Kaplan, `\href{https://www.theatlantic.com/magazine/archive/1994/02/the-coming-anarchy/304670/}{The Coming Anarchy: How scarcity, crime, overpopulation, tribalism, and disease are rapidly destroying the social fabric of our planet}.' \textit{The Atlantic,} February 1994.
\end{itemize}

\hline % ============================================================

\subsection{Civil wars II}\label{cw2}

Understanding the role of grievances in the outbreak of civil wars. Modern grievance-based explanations highlight the role of political inequality (especially along ethnic lines) in increasing the risk of war onset. A new consensus includes both motivation and opportunity factors.

\subsubsection*{Lecture}

\begin{itemize}
\setlength\itemsep{0pt}
\item Chps 1 \& 2 in Lars-Erik Cederman, Kristian Skrede Gleditsch \& Halvard Buhaug, \href{https://doi.org/10.1017/CBO9781139084161}{\textit{Inequality, grievances, and civil war}}. Cambridge UP, 2013: pp. 1--29.
\end{itemize}

\subsubsection*{Seminar}

\begin{itemize}
\setlength\itemsep{0pt}
\item Anand Gopal, `\href{https://www.newyorker.com/magazine/2021/09/13/the-other-afghan-women}{The other Afghan women}.' \textit{The New Yorker,} 06/09/2021.
\end{itemize}

\hline % ============================================================

\subsection{Wartime violence}\label{wartime_violence}

The repertoire of violence during wars. Types of violence and definitions. Focus on violence against civilians. Causes and dynamics. Ethnic violence and genocide.

\subsubsection*{Lecture}

\begin{itemize}
\setlength\itemsep{0pt}
\item Benjamin Valentino, `\href{https://doi.org/10.1146/annurev-polisci-082112-141937}{Why we kill: The political science of political violence against civilians}.' \textit{Annual Review of Political Science} 17: 89--103, 2014.
\end{itemize}

\subsubsection*{Seminar}

\begin{itemize}
\setlength\itemsep{0pt}
\item Joshua Yaffa, `\href{https://www.newyorker.com/magazine/2023/02/06/the-hunt-for-russian-collaborators-in-ukraine}{The Hunt for Russian Collaborators in Ukraine}.' \textit{The New Yorker}, 06/02/2023.
\end{itemize}

\hline % ============================================================

\subsection{Non-state armed actors and civilians}\label{rebels}

What happens behind the fronts? Rebel governance and recruitment. How do armed groups control the civilian population? Wartime social processes.

\subsubsection*{Lecture}

\begin{itemize}
\setlength\itemsep{0pt}
\item Chapter 1 (`Introduction') in Ana Arjona, Nelson Kasfir \& Zachariah Mampilly, \href{https://doi.org/10.1017/CBO9781316182468}{\textit{Rebel governance in civil war}}. Cambridge UP, 2015: pp. 1--20.
\end{itemize}

\subsubsection*{Seminar}

\begin{itemize}
\setlength\itemsep{-5pt}
\item Jon Lee Anderson, `\href{https://www.newyorker.com/magazine/2023/07/24/haiti-held-hostage}{Haiti held hostage}.' \textit{The New Yorker}, 24/07/2023.
% \item Joshua Yaffa, `\href{https://www.newyorker.com/magazine/2022/05/23/a-ukrainian-city-under-a-violent-new-regime}{A Ukrainian city under a violent new regime}.' \textit{The New Yorker}, 16/05/2022.
\end{itemize}


\hline % ============================================================

\subsection{Terrorism}\label{terrorism}

Despite its relevance, terrorism is usually misunderstood. Terrorist actions and terrorist groups. Dynamics and causes. Suicide terrorism.

\subsubsection*{Lecture}

\begin{itemize}
\setlength\itemsep{0pt}
\item Luis de la Calle \& Ignacio Sánchez-Cuenca, `\href{https://doi.org/10.1177/0032329211415506}{What we talk about when we talk about terrorism}.' \textit{Politics \& Society} 39(3): 451--472, 2011.
\end{itemize}

\subsubsection*{Seminar}

\begin{itemize}
\setlength\itemsep{0pt}
\item Lawrence Wright, `\href{https://www.newyorker.com/magazine/2006/09/11/the-master-plan}{The master plan}'. \textit{The New Yorker}, 11/09/2006.
% \item David Pilling, `\href{https://www.ft.com/content/744bea94-3b18-47d5-8e53-e87ab9efef9a}{Niger: the west's bulwark against jihadis and Russian influence in Africa}.' \textit{Finantial Times,} 07/07/2022.
% \item Elisabeth Zarofsky, `\href{https://www.newyorker.com/news/letter-from-europe/a-terrorist-attack-on-yom-kippur-in-halle-germany}{A terrorist attack on Yom Kippur in Halle, Germany}. \textit{The New Yorker}, 13/10/2019.
\end{itemize}

\hline % ============================================================

\subsection{Postwar politics and prevention}\label{postwar}

Strategies to deal with conflict-ridden countries. Effects of power-sharing, regional autonomy, and secession. Postwar democratization. Combatatant demobilization.

\subsubsection*{Lecture}

\begin{itemize}
\setlength\itemsep{0pt}
\item Hanna Leonardsson \& Gustav Rudd, `\href{https://doi.org/10.1080/01436597.2015.1029905}{The ‘local turn’ in peacebuilding: a literature review of effective and emancipatory local peacebuilding}.' \textit{Third World Quarterly} 36(5): 825-839, 2015.
\item Lars-Erik Cederman, Simon Hug, \& Julian Wucherpfennig, \href{https://doi.org/10.1017/9781108284639}{\textit{Sharing Power, Securing Peace?: Ethnic Inclusion and Civil War}}. Cambridge UP, 2022, chapters 2 \& 3.
\end{itemize}

\subsubsection*{Seminar}

\begin{itemize}
\setlength\itemsep{0pt}
% \item Jon Lee Anderson, `\href{https://www.newyorker.com/magazine/2022/02/28/the-taliban-confront-the-realities-of-power-afghanistan}{The Taliban confront the realities of power}.' \textit{The New Yorker,} 21/02/2022.
\item Etgar Keret, `\href{https://www.newyorker.com/books/page-turner/israels-other-war}{Israel's other war}.' \textit{The New Yorker,} 25/07/2014.
\item Dennis Ross, `\href{https://www.foreignaffairs.com/israel/hamas-what-israel-must-do}{What Israel must do: Disarming Hamas will be costly but essential for peace}.' \textit{Foreign Affairs,} 11/10/2023.
\end{itemize}


\hline % ============================================================

\subsection{Transitional Justice}\label{tj}

The emergence of international norms on accountability for human rights violations. Types of transitional justice policies, determinants and consequences.

\subsubsection*{Lecture}

\begin{itemize}
\setlength\itemsep{0pt}
\item Kathryn Sikkink and Hun Joon Kim, `\href{https://doi.org/10.1146/annurev-lawsocsci-102612-133956}{The Justice Cascade: The Origins and Effectiveness of Prosecutions of Human Rights Violations}.' \textit{Annual Review of Law and Social Science} 9: 269--285, 2013.
\end{itemize}

\subsubsection*{Seminar}

\begin{itemize}
\setlength\itemsep{0pt}
\item Masha Gessen, `\href{https://www.newyorker.com/magazine/2022/08/08/the-prosecution-of-russian-war-crimes-in-ukraine}{The Prosecution of Russian War Crimes in Ukraine}.' \textit{The New Yorker,} 01/08/2022.
\end{itemize}

\hline % ============================================================

\subsection{Legacies and consequences of war}\label{legacies}

Wars, especially wartime violence, transform countries and societies fundamentally. Consequences of civil wars on the civilian population. Long-term legacies on preferences.

\subsubsection*{Lecture}

\begin{itemize}
\setlength\itemsep{0pt}
\item Jacob Walden and Yuri M. Zhukov, `Historical legacies of political violence.' In \textit{Oxford Research Encyclopedia of Politics}, 2020.
\end{itemize}

\subsubsection*{Seminar}

\begin{itemize}
\setlength\itemsep{0pt}
\item Benjamin Wallace-Wells, `\href{https://www.newyorker.com/magazine/2017/12/04/the-fight-over-virginias-confederate-monuments}{The Fight Over Virginia’s Confederate Monuments}.' \textit{The New Yorker,} 27/11/2017.
\end{itemize}


\hline % ============================================================

\subsection{Wrap-up} \label{wrap_up}

\subsubsection*{Lecture}

\begin{itemize}
\setlength\itemsep{0pt}
\item Stathis Kalyvas (2019), `The landscape of political violence.' \textit{The Oxford Handbook of Terrorism}, chapter 2.
% \item Stathis Kalyvas \& Scott Straus (2020), \href{https://doi.org/10.1177/2633002420972955}{Stathis Kalyvas on 20 years of studying political violence}.' \textit{Violence: An International Journal} 1(2): 389--407.
\end{itemize}

\subsubsection*{Seminar}

Extra presentations (\textbf{max 2, 25\% penalty}).

\end{document}
