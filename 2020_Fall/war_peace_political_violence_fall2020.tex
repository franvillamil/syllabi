\documentclass[12pt, a4paper]{article}
\usepackage[margin = 2cm]{geometry}
\usepackage{graphicx}
\usepackage[english]{babel}
\usepackage[utf8]{inputenc}
\usepackage[colorlinks = TRUE]{hyperref}
\usepackage{setspace}
\setstretch{1.25}
\renewcommand*\rmdefault{ppl}


\usepackage[]{titlesec}
    \titleformat*{\section}{\large\bf}
    \titleformat*{\subsection}{\normalsize\bf}
    \titleformat*{\subsubsection}{\normalsize\it}

%%%%%%%%%%%%%%%%%%%%%%%%%%%%%%%%%%%%
\begin{document}
\begin{center}
{\Large War, peace, and political violence}\\\vspace{10pt}
Bachelor in History and Politics\\
Universidad Carlos III de Madrid\\
Fall 2020\\
\end{center}

\vspace{20pt}

\begin{minipage}{0.49\textwidth}
\centering
Francisco Villamil\\
\href{francisco.villamil@uc3m.es}{francisco.villamil@uc3m.es}\\
Office: 18.2.E05\\
Office hours: by appointment
\end{minipage}\hfill
\begin{minipage}{0.49\textwidth}
\centering
Lecture (online)\\Tuesday 09:00--10:30h\\\vspace{5pt}
Seminar (Room 17.0.05)\\Thursday 10:45--12:15h\\
\end{minipage}


\vspace{10pt}
\section{Description}

This course provides an overview over a wide range of topics in conflict research, including inter-state wars, civil wars, and causes and dynamics of political violence.
Its main goal is to provide students with the conceptual and theoretical tools to think analytically about conflicts and political violence. Some of the questions we will explore are: Why do countries fight each other? How do changes in the international system impact conflicts across the world? What explains the outbreak of civil wars? Why and how civilians are killed during wars? What are the long-term consequences of conflicts?

\section{Requirements}

We meet twice a week. In the lectures, we will review the main debates in each topic. Each lecture has one reading assigned, usually a research article, that covers part or most of what we will talk about. Reading it is not mandatory, but recommended, either before or after the lecture. In each seminar, we will discuss a reading related to the lecture. These readings, which are \textbf{mandatory}, are shorter and lighter than the ones assigned to the lectures, and are meant to reflect or expand on the topic covered each week.

\section{Materials}

You can find all the reading materials in \textit{Aula Global}.

\section{Assessment}

Students will be evaluated based on three different assignments:

\subsection*{Response papers (20\%)}

Each student will have to choose \textbf{two} topics/weeks, and write a 1- to 2-page critical summary of the weekly readings. Response papers (10\% of the grade each) will be due on the seminar day chosen, which can be selected in the Doodle link below (max 3 people per week: first come, first served):

\begin{itemize}
\setlength\itemsep{0pt}
\item[] \href{https://doodle.com/poll/wf3q29s9tzd2expy}{https://doodle.com/poll/wf3q29s9tzd2expy}
\end{itemize}

\subsection*{Presentation (20\%)}

In the last two seminar days (December 10th and 17th), in groups of 2-3 people, students will have to give a 15-min presentation, which will be followed by a 5- to 10-min Q\&A. Grading will be based on both the presentation and participation during the Q\&A. We can discuss in class alternatives but, in principle, there are two options:

\begin{itemize}
\setlength\itemsep{0pt}
\item[a)] An overview of a single conflict, reflecting on one or more topics covered in class (for example: `Violence against civilians in Syria')
\item[b)] A topic and its relevance in modern or historical times, expanding what we covered in class (for example: `Nationalism and conflict in the 21st Century')
\end{itemize}

\subsection*{Final take-home exam (60\%)}

A final take-home exam, which students will have 24h to complete. Its goal is to evaluate how well students understood the main concepts and ideas. The date is open for discussion.

\section{Course outline}

\subsection*{15 Sept - Introduction}

Presentation. Course structure and organizational issues. Introduction: what is political violence and what are we going to talk about?

\begin{itemize}
\setlength\itemsep{-5pt}
\item \textit{No readings}
\end{itemize}

\subsubsection*{Seminar (17 Sept)}

\begin{itemize}
\setlength\itemsep{-5pt}
\item Luke Mogelson, `The militias against masks.' \textit{The New Yorker}, 24/08/2020.
\item[] \href{https://www.newyorker.com/magazine/2020/08/24/the-militias-against-masks}{(www.newyorker.com/magazine/2020/08/24/the-militias-against-masks)}
\end{itemize}

\subsection*{22 Sept - Concepts}

The typologies of conflict and violence. Why violence is not enough to identify a war. The problem of defining the start and end of conflicts. Basic ideas: actors involved, types of violence, objectives, methods, non-fatal violence and repression.

\begin{itemize}
\setlength\itemsep{0pt}
\item Stathis Kalyvas, `The ontology of political violence.' \textit{Perspectives on Politics} 1(3): 475--494, 2003.
\end{itemize}

\subsubsection*{Seminar (24 Sept)}

\begin{itemize}
\setlength\itemsep{-5pt}
\item Elad Uzan, `How do we end the never-ending wars?' \textit{Boston Review,} 02/10/2019.
\item[] \href{https://bostonreview.net/war-security/elad-uzan-how-do-we-end-never-ending-wars}{bostonreview.net/war-security/elad-uzan-how-do-we-end-never-ending-wars}
\end{itemize}

\subsection*{29 Sept - Classic wars}

Inter-state wars and classical explanations of warfare. The three visions in IR. Decent decline of inter-state war, the democratic peace and economic interdependences.

\begin{itemize}
\setlength\itemsep{0pt}
\item Chapter 1 in Kalevi J Holsti, \textit{Peace and War: Armed Conflicts and International Order, 1648-1989.} Cambridge UP, 1991: pp. 1--24.
\end{itemize}

\subsubsection*{Seminar (1 Oct)}

\begin{itemize}
\setlength\itemsep{-5pt}
\item Graham Allison, `The Thucydides' Trap: Are the U.S. and China headed for war?' \textit{The Atlantic,} 24/09/2015.
\item[] \href{https://www.theatlantic.com/international/archive/2015/09/united-states-china-war-thucydides-trap/406756/}{www.theatlantic.com/international/archive/2015/09/united-states-china-war-thucydides-trap/406756/}
% \item {\color{red}{\textbf{TBD}}}
% \item[\textbf{???}] Chapters 1 \& 8 in Kenneth Waltz, \textit{Man, the State and War.} Columbia UP, 1959: pp. XX--XX.
\end{itemize}

\subsection*{6 Oct - State-building and war}

State-building and war: origins of the state, role of international conflict in the creation of states and vice-versa, role of violence in the internal history of states. The new international order after 1648.

\begin{itemize}
\setlength\itemsep{0pt}
\item Chapter 1 in Charles Tilly, \textit{Coercion, capital, and European states, AD 990-1992.} Blackwell Publishing, 1990: pp. 1--37.
% \item Charles Tilly, `War making and state making as organized crime.' In \textit{Bringing the State Back In} (ed. P. Evans, D. Rueschemyer \& T. Skocpol). Cambridge UP, 1985: pp. 169--187.
\end{itemize}

\subsubsection*{Seminar (8 Oct)}

\begin{itemize}
\setlength\itemsep{0pt}
\item Chp 5 in James C Scott, \textit{Against the Grain: A Deep History of the Earliest States.} Yale UP, 2017: pp. 150--182.
\end{itemize}

\subsection*{13 Oct - The rise of Nation-states}

The French Revolution not only changed peaceful politics, but also war. The development of nationalisms and its relationship with political violence. 'The people' as a source of political legitimacy, wars of independencia and civil wars. Recruitment and mobilization capacity of the new Nation-state.

\begin{itemize}
\setlength\itemsep{0pt}
\item Lars-Erik Cederman, T. Camber Warren \& Didier Sornette, `Testing Clausewitz: Nationalism, Mass Mobilization, and the Severity of War.' \textit{International Organization} 65(4): 605--638, 2011.
\end{itemize}

\subsubsection*{Seminar (15 Oct)}

\begin{itemize}
\setlength\itemsep{0pt}
\item Chp 1 in Michael Mann, \textit{The dark side of democracy: Explaining ethnic cleansing.} Cambridge UP, 2004: 1--33.
\end{itemize}

\subsection*{20 Oct - Civil wars I: greed-based explanations}

Basic concepts and types of civil wars. After 1990, there is a deep increase in the outbreak of civil wars. What used to be explained as popular revolutions, now is seen as a problem of anarchy, looting, and greed.

\begin{itemize}
\setlength\itemsep{0pt}
\item James Fearon \& David Laitin, `Ethnicity, insurgency, and civil war.' \textit{American Political Science Review} 97(1): 75--90, 2003.
\end{itemize}

\subsubsection*{Seminar (22 Oct)}

\begin{itemize}
\setlength\itemsep{-5pt}
\item Robert D Kaplan, `The Coming Anarchy: How scarcity, crime, overpopulation, tribalism, and disease are rapidly destroying the social fabric of our planet.' \textit{The Atlantic,} February 1994.
\item[] \href{https://www.theatlantic.com/magazine/archive/1994/02/the-coming-anarchy/304670/}{www.theatlantic.com/magazine/archive/1994/02/the-coming-anarchy/304670/}
\end{itemize}

\subsection*{27 Oct - Civil wars II: grievance-based explanations}

Before 1990, civil wars were usually explained as popular revolutions caused by resentment and inequality. This explanation was sidelined after the `New Wars' of the 1990s. Modern grievance-based explanations highlight the role of political inequality (especially along ethnic lines) in increasing the risk of war onset. A new consensus includes both motivation and opportunity factors.

\begin{itemize}
\setlength\itemsep{0pt}
\item Chps 1 \& 2 in Lars-Erik Cederman, Kristian Skrede Gleditsch \& Halvard Buhaug, \textit{Inequality, grievances, and civil war.} Cambridge UP, 2013: pp. 1--29.
\end{itemize}

\subsubsection*{Seminar (29 Oct)}

\begin{itemize}
\setlength\itemsep{-5pt}
\item Laia Balcells, `A way out of Spain's Catalan crisis.' \textit{Foreign Affairs,} 27/11/2019.
\item[] \href{https://www.foreignaffairs.com/articles/europe/2019-11-27/way-out-spains-catalan-crisis}{www.foreignaffairs.com/articles/europe/2019-11-27/way-out-spains-catalan-crisis}
\item[]
\item Lars-Erik Cederman, `Blood for soil: The fatal temptations of ethnic politics.' \textit{Foreign Affairs,} March/April 2019.
\item[] \href{https://www.foreignaffairs.com/articles/2019-02-12/blood-soil}{www.foreignaffairs.com/articles/2019-02-12/blood-soil}
\end{itemize}

\subsection*{3 Nov - Violence during war}

The repertoire of violence during wars. Types of violence and definitions. Focus on violence against civilians. Causes and dynamics. Ethnic violence and genocide.

\begin{itemize}
\setlength\itemsep{0pt}
\item Benjamin Valentino, `Why we kill: The political science of political violence against civilians.' \textit{Annual Review of Political Science} 17: 89--103, 2014.
\end{itemize}

\subsubsection*{Seminar (5 Nov)}

\begin{itemize}
\setlength\itemsep{0pt}
\item Chapter 1 in Beatriz Manz, \textit{Paradise in ashes: A Guatemalan journey of courage, terror, and hope.} U of California Press, 2004: pp. 1--32.
\end{itemize}

\subsection*{10 Nov - Non-state armed groups}

What happens behind the fronts? Rebel governance and recruitment. How do armed groups control the civilian population? Wartime social processes.

\begin{itemize}
\setlength\itemsep{0pt}
\item Chapter 1 (`Introduction') in Ana Arjona, Nelson Kasfir \& Zachariah Mampilly, \textit{Rebel governance in civil war.} Cambridge UP, 2015: pp. 1--20.
% \item \textbf{???} Or Chp 14 (Conclusion)? Or Arjona 2014 JCR, Wartime institutions: A research agenda???
\end{itemize}

\subsubsection*{Seminar (12 Nov)}

\begin{itemize}
\setlength\itemsep{-5pt}
\item Mara Revkin, `ISIS' social contract: What the Islamic State offers civilians.' \textit{Foreign Affairs,} 01/10/2016.
\item[] \href{https://www.foreignaffairs.com/articles/syria/2016-01-10/isis-social-contract}{www.foreignaffairs.com/articles/syria/2016-01-10/isis-social-contract}
\end{itemize}

\subsection*{17 Nov - Terrorism}

Despite its relevance, terrorism is usually misunderstood. Terrorist actions and terrorist groups. Dynamics and causes. Suicide terrorism.

\begin{itemize}
\setlength\itemsep{0pt}
\item Luis de la Calle \& Ignacio Sánchez-Cuenca, `What we talk about when we talk about terrorism.' \textit{Politics \& Society} 39(3): 451--472, 2011.
\end{itemize}

\subsubsection*{Seminar (19 Nov)}

\begin{itemize}
\setlength\itemsep{0pt}
\item Intro \& chp 1 in Ignacio Sánchez-Cuenca, \textit{The Historical Roots of Political Violence: Revolutionary Terrorism in Affluent Countries.} Cambridge UP, 2019: 1--31.
\end{itemize}

\subsection*{24 Nov - Legacies of conflict}

Wars, especially wartime violence, transform countries and societies fundamentally. Consequences of civil wars on the civilian population. Long-term legacies on preferences.

\begin{itemize}
\setlength\itemsep{0pt}
\item Elisabeth J Wood, `The social processes of civil war: The wartime transformation of social networks.' \textit{Annual Review of Political Science} 11: 539--561, 2008.
\end{itemize}

\subsubsection*{Seminar (26 Nov)}

\begin{itemize}
\setlength\itemsep{0pt}
\item pp. 62--81 in William Finnegan, \textit{A complicated war: The harrowing of Mozambique.} U of California Press, 1992.
% \item[\textbf{????}] https://www.newyorker.com/magazine/2020/01/13/why-a-champion-of-reparative-justice-turned-on-the-cause
\end{itemize}

\subsection*{1 Dec - Peacebuilding and postwar politics}

Strategies to deal with conflict-ridden contries. Effects of power-sharing, regional autonomy, and secession. Postwar democratization. Combatatant demobilization. Transitional justice.

\begin{itemize}
\setlength\itemsep{0pt}
\item Virginia P Fortna \& Lise M Howard, `Pitfalls and prospects in the peacekeeping literature.' \textit{Annual Review of Political Science} 11: 283--301, 2008.
\end{itemize}

\subsubsection*{Seminar (3 Dec)}

\begin{itemize}
\setlength\itemsep{-5pt}
\item Robert D Kaplan, `Buddha's savage peace.' \textit{The Atlantic,} September 2009.
\item[] \href{https://www.theatlantic.com/magazine/archive/2009/09/buddhas-savage-peace/307620/}{www.theatlantic.com/magazine/archive/2009/09/buddhas-savage-peace/307620/}
\end{itemize}

\subsection*{8 Dec - NO LECTURE}

\textit{(Inmaculada Concepción.)}

\subsubsection*{Seminar (10 Dec)}

Presentations, first day.

\subsection*{15 Dec - A decline in violence?}

Discussing the decline in violence, particularly in terms of inter-state conflict. Wrap-up of the course.

\begin{itemize}
\setlength\itemsep{0pt}
\item Chapter 10 in Steven Pinker, \textit{The better angels of our nature: Why violence has declined.} Penguin, 2012: 671--696.
\item Lars-Erik Cederman, `Back to Kant: Reinterpreting the Democratic Peace as a Macrohistorical Learning Process.' \textit{American Political Science Review} 95(1): 15--31, 2001.
\end{itemize}

\subsubsection*{Seminar (17 Dec)}

Presentations, second day.

\end{document}
