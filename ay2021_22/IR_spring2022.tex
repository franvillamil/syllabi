\documentclass[12pt, a4paper]{article}
\usepackage[margin = 1in]{geometry}
\usepackage{graphicx}
\usepackage[english]{babel}
\usepackage[utf8]{inputenc}
\usepackage[colorlinks = TRUE]{hyperref}
\usepackage{setspace}
\setstretch{1.25}
%\renewcommand*\rmdefault{ppl}


\usepackage[]{titlesec}
    \titleformat*{\section}{\large\bf}
    \titleformat*{\subsection}{\normalsize\bf}
    \titleformat*{\subsubsection}{\normalsize\it}

%%%%%%%%%%%%%%%%%%%%%%%%%%%%%%%%%%%%
\begin{document}
\begin{center}
{\Large \textbf{International Relations}}\\\vspace{10pt}
Master in Social Sciences\\
Carlos III -- Juan March Institute of Social Sciences\\
Spring 2022\\
\end{center}

\vspace{20pt}

\begin{minipage}{0.48\textwidth}
\textbf{Dr. Iasmin Goes}\\
\textbf{Email:} \url{igoesara@clio.uc3m.es}\\
\textbf{Office:} 18.2.E.10\\
\textbf{Office hours:} by appointment
\end{minipage}\hfill
\begin{minipage}{0.48\textwidth}
% \centering
\textbf{Dr. Francisco Villamil}\\
\textbf{Email:} \url{francisco.villamil@uc3m.es}\\
\textbf{Office:} 18.2.E05\\
\textbf{Office hours:} by appointment
\end{minipage}

\vspace{10pt}
\subsection*{Meeting Time and Location:}

Thursdays 10 am -- 1 pm, Aula 18.1.A01

\section{Description}

This course covers the two main subfields of international relations: international security and international political economy.

The international political economy sessions will cover four topics: sovereign debt, trade, exchange rates, and investment. We will discuss how interest groups, voters, bureaucrats, policymakers, firms, and international organizations interact to shape policy outcomes pertaining to these four topics. For example, when do economic elites push for capital liberalization? Why are migrant remittances associated with fixed exchange rates? The goal is to understand how such policies come about, but also how they might affect other issue areas in political science.

The conflict part covers several phenomena related to political violence, including interstate wars, civil wars, wartime violence, or armed group behavior. Some of the questions we will explore are: Why do countries fight each other? How do changes in the international system impact conflicts across the world? What explains the outbreak of civil wars? Why and how civilians are killed during wars? What are the long-term consequences of conflicts? What is terrorism? We will cover both the conceptual and theoretical reasoning behind these questions as well as the methodological approaches used to answer them empirically.

\section{Requirements and assessment}

\begin{itemize}
\item 20\% -- Class attendance, participation, and presentation
\item 80\% -- Research paper
\end{itemize}

You will develop a research project related to one or more topics discussed in class. You will present this project to your instructors and colleagues on \textbf{Thursday, May 5}, and submit a research paper (around 4,000 words, excluding references, tables, and figures) by the end of the semester (date TBD). We will talk about the project, presentation, and paper in more detail over the course of the semester.

\section{Materials}

You can find all the reading materials in \textit{Aula Global}.
Each week, the reading list also includes optional texts that provide some background information, in case you are not familiar with the core concepts or just want to brush up on your IR knowledge.
They are also meant for those of you who are interested in some particular topic, in case you want to dig deeper into it.

\section{Course outline}

\subsection*{Feb 3 - Session 1: Introduction to IR (Iasmin \& Fran)}

\begin{itemize}
\item Dan Reiter. 2015. ``Should We Leave Behind the Subfield of International Relations?'' \textit{Annnual Review of Political Science} 18: 481--499.
\end{itemize}

\subsection*{Feb 10 - Session 2: Sovereign debt (Iasmin)}

\noindent \textbf{Required readings}

\begin{itemize}
\item K. Amber Curtis, Joseph Jupille and David Leblang. 2014. ``Iceland on the Rocks: The Mass Political Economy of Sovereign Debt Resettlement.'' \emph{International Organization} 68(3):721--740.
\item Patrick E. Shea and Paul Poast. 2018. ``War and Default.'' \emph{Journal of Conflict Resolution} 62(9):1876-1904.
\end{itemize}

\noindent \textbf{Recommended/extra}

\begin{itemize}
\item Michael Tomz and Mark L. J. Wright. 2013. ``Empirical Research on Sovereign Debt and Default.'' \emph{Annual Review of Economics} 5:247--272.
\end{itemize}
% https://mobile.twitter.com/profpaulpoast/status/1169630321291210753

\subsection*{Feb 17 - Session 3: Introduction, interstate wars and classic studies (Fran)}

\noindent \textbf{Required readings}

\begin{itemize}
  \item Kenneth Waltz, \textit{Man, the State and War.} Columbia University Press, 1959. Chapters 2, 4, and 6.
  \item Kalevi J Holsti, \textit{Peace and War: Armed Conflicts and International Order, 1648-1989.} Cambridge University Press, 1991. Chapter 1: pp. 1--24.
\end{itemize}

\noindent \textbf{Recommended/extra}

\begin{itemize}
  \item Jack S Levy, `War and Peace.' In: \textit{The Handbook of International Relations} (ed. W. Carlsnaes, T. Risse and B. Simmons). Sage, 2013: pp. 581--606.
  \item Rest of chapters in Waltz (1959) and Holsti (1991).
\end{itemize}

\subsection*{Feb 24 - Session 4: State-building, nationalism, and war (Fran)}

\noindent \textbf{Required readings}

\begin{itemize}
  \item Charles Tilly, `War making and state making as organized crime.' In \textit{Bringing the State Back In} (ed. P. Evans, D. Rueschemyer \& T. Skocpol). Cambridge UP, 1985: pp. 169--187.
  \item Andreas Wimmer and Brian Min, `From empire to nation-state. Explaining wars in the modern world, 1816-2001.' \textit{American Sociological Review} 71: 867--897, 2006.
  \item Lars-Erik Cederman, T. Camber Warren \& Didier Sornette, `Testing Clausewitz: Nationalism, Mass Mobilization, and the Severity of War.' \textit{International Organization} 65(4): 605--638, 2011.
\end{itemize}

\noindent \textbf{Recommended/extra}

\begin{itemize}
  \item Charles Tilly, \textit{Coercion, capital, and European states, AD 990-1992.} Blackwell Publishing, 1990. Chapter 1: pp. 1--37.
  \item Kalevi J Holsti, \textit{Peace and War: Armed Conflicts and International Order, 1648-1989.} Cambridge University Press, 1991. Chapters 2 (pp. 25-42) and 7 (pp. 138--174).
  \item James C Scott, \textit{Against the Grain: A Deep History of the Earliest States.} Yale University Press, 2016. Chapter 5: pp. 150--182.
  \item Michael Mann, \textit{The dark side of democracy: Explaining ethnic cleansing.} Cambridge UP, 2004. Chapter 1: pp 1--33.
\end{itemize}

\subsection*{March 3 - Session 5: Free trade vs. protectionism (Iasmin)}

\noindent \textbf{Required readings}

\begin{itemize}
\item Anna Maria Mayda and Dani Rodrik. 2005. ``Why Are Some People (and Countries) More Protectionist Than Others?'' \emph{European Economic Review} 49(6):1393--1430.
\item Timm Betz, David Fortunato, and Diana Z. O'Brien. 2021. ``Women's Descriptive Representation and Gendered Import Tax Discrimination.'' \emph{American Political Science Review} 115(1): 307--315.
\item Emilie M. Hafner-Burton. 2005. ``Trading Human Rights: How Preferential Trade Agreements Influence Government Repression.'' \emph{International Organization} 59(3): 593--629.
\end{itemize}

\noindent \textbf{Recommended/extra}

\begin{itemize}
\item K. Alec Chrystal, Cletus C. Coughlin and Geoffrey E. Wood. 1988. ``Protectionist Trade Policies: A Survey of Theory, Evidence and Rationale.'' \emph{Federal Reserve Bank of St. Louis Review} 70(1):12--29.
\item Helen V. Milner and Keiko Kubota. 2005. ``Why the Move to Free Trade? Democracy and Trade Policy in the Developing Countries.'' \emph{International Organization} 59(1):107--143.
\end{itemize}

\subsection*{March 10 - Session 6: Exchange rates and monetary unions (Iasmin)}

\noindent \textbf{Required readings}

\begin{itemize}
\item Charles Wyplosz. 2000. ``EMU: Why and How It Might Happen.'' In: Jeffry A. Frieden and David A. Lake (eds.). \emph{International Political Economy: Perspectives on Global Power and Wealth.} London and New York: Routledge, pp. 270--279.
\item Jeffry Frieden. 2015. ``European Monetary Integration: From Bretton Woods to the Euro and Beyond.'' \emph{Currency Politics: The Political Economy of Exchange Rate Policy.} Princeton: Princeton University Press. pp. 137--185.
\item David Andrew Singer. 2010. ``Migrant Remittances and Exchange Rate Regimes in the Developing World.'' \emph{American Political Science Review} 104(2):307--323.
\end{itemize}

\noindent \textbf{Recommended/extra}

\begin{itemize}
\item J. Lawrence Broz and Jeffry Frieden. 2001. ``The Political Economy of International Monetary Relations.'' \emph{Annual Review of Political Science} 4:317--343.
\item J. Lawrence Broz. 2000. ``The Domestic Politics of International Monetary Order: The Gold Standard.'' In: Jeffry A. Frieden and David A. Lake. \emph{International Political Economy: Perspectives on Global Power and Wealth.} London and New York: Routledge, pp. 199--219.
\end{itemize}

\subsection*{March 17 - Session 7: Civil wars (Fran)}

\noindent \textbf{Required readings}

\begin{itemize}
  \item Stathis Kalyvas, \textit{The Logic of Violence in Civil War.} Cambridge University Press, 2006. Chapter 1: pp. 16-31.
  \item James Fearon \& David Laitin, `Ethnicity, insurgency, and civil war.' \textit{American Political Science Review} 97(1): 75--90, 2003.
  \item Lars-Erik Cederman, Kristian Skrede Gleditsch \& Halvard Buhaug, \textit{Inequality, grievances, and civil war.} Cambridge UP, 2013: chapters 1--4 (pp. 1--92). (Read at least chapters 1 and 2)
\end{itemize}

\noindent \textbf{Recommended/extra}

\begin{itemize}
  \item Christopher Blattman and Edward Miguel (2010) Civil War. \textit{Journal of Economic Literature} 48(1): 3-57.
  \item Paul Collier and Anke Hoeffler (2004) Greed and Grievance in Civil Wars. Oxford Economic Papers 56, pp. 563-595.
  \item Barry Posen (1993) The Security Dilemma and Ethnic Conflict. \textit{Survival} 35:1: 27-47.
  \item Chaim Kaufmann (1996) Possible and Impossible Solutions to Ethnic Civil Wars. \textit{International Security} 20:4: 136-175.
\end{itemize}

\subsection*{March 24 - Session 8: Wartime violence (Fran)}

\noindent \textbf{Required readings}

\begin{itemize}
  \item Stathis Kalyvas, `The ontology of political violence.' \textit{Perspectives on Politics} 1(3): 475--494, 2003.
  \item Benjamin Valentino, `Why we kill: The political science of political violence against civilians.' \textit{Annual Review of Political Science} 17: 89--103, 2014.
  \item Laia Balcells and Jessica A Stanton, `Violence against civilians during armed conflict: Moving beyond the macro- and micro-level Divide.' \textit{Annual Review of Political Science} 24: 45--69.
\end{itemize}

\noindent \textbf{Recommended/extra}

\begin{itemize}
  \item Jacob Walden and Yuri M Zhukov (2020) Historical legacies of political violence. In: \textit{Oxford Research Encyclopedia of Politics}, Oxford University Press.
  \item Nils B Weidmann (2011) Violence "from above" or "from below"? The role of ethnicity in Bosnia's civil war. \textit{Journal of Politics} 73(4): 1178-1190.
  \item Ragnhild Nordås and Dara Kay Cohen (2021) Conflict-related sexual violence. \textit{Annual Review of Political Science} 24: 193--211.
\end{itemize}

\subsection*{March 31 - Session 9: Capital flows and portfolio investment (Iasmin)}

\noindent \textbf{Required readings}

\begin{itemize}
\item Layna Mosley. 2000. ``Room to Move: International Financial Markets and National Welfare States.'' \emph{International Organization} 54(4):737--773.
\item Daniela Campello. 2014. ``The Politics of Financial Booms and Crises: Evidence from Latin America.'' \emph{Comparative Political Studies} 47(2): 260--286.
\end{itemize}

\noindent \textbf{Recommended/extra}

\begin{itemize}
\item Dennis Quinn and Carla Inclán. 1997. ``The Origins of Financial Openness: A Study of Current and Capital Account Liberalization.'' \emph{American Journal of Political Science} 41(3):771--813.
\item Christopher J. Neely. 1999. ``An Introduction to Capital Controls.'' \emph{Federal Reserve Bank of St. Louis Review}: 13--30.
\end{itemize}

\subsection*{April 7 - Session 10: Foreign direct investment and multinational corporations (Iasmin)}

\noindent \textbf{Required readings}

\begin{itemize}
\item Quan Li and Adam Resnick. 2003. ``Reversal of Fortunes: Democratic Institutions and Foreign Direct Investment Flows to Developing Countries.'' \emph{International Organization} 57(1):175--211.
\item Sarah Bauerle Danzman. 2020. ``Foreign Direct Investment Policy, Domestic Firms, and Financial Constraints.'' \emph{Business and Politics} 22(2):279--306.
\end{itemize}

\noindent \textbf{Recommended/extra}

\begin{itemize}
\item John Ahlquist. 2006. ``Economic Policy, Institutions, and Capital Flows: Portfolio and Direct Investment Flows in Developing Countries.'' \emph{International Studies Quarterly} 50(3):681--704.
\item Nathan M. Jensen, Glen Biglaiser, Quan Li, Edmund Malesky, Pablo Pinto, Santiago Pinto and Joseph Staats. 2012. ``Introduction: Multinational Corporations and Governments.'' \emph{Politics and Foreign Direct Investment.} Ann Arbor: University of Michigan Press, pp. 1--26.
\end{itemize}

\subsection*{April 21 - Session 11: Terrorism and armed groups (Fran)}

\noindent \textbf{Required readings}

\begin{itemize}
  \item Luis de la Calle and Ignacio Sánchez-Cuenca, \textit{Underground violence: A theory of terrorism.} Manuscript, 2021. Chapters Introduction, 1 and 2 (pp. 2-75).
  \item Ana Arjona, Nelson Kasfir and Zachariah Mampilly, \textit{Rebel governance in civil war.} Cambridge University Press, 2015. Chapter 1 (pp. 1-20).
\end{itemize}

\noindent \textbf{Recommended/extra}

\begin{itemize}
  \item Virginia P. Fortna (2015) Do Terrorists Win? Rebels’ Use of Terrorism and Civil War Outcomes. \textit{International Organization} 69(3): 519-556.
  \item Barbara F. Walter (2017) The Extremist’s Advantage in Civil Wars. \textit{International Security} 42(2): 7-39.
  \item Macartan Humphreys and Jeremy M Weinstein (2006) Handling and Manhandling Civilians in Civil War. \textit{American Political Science Review} 100(3): 429-447.
  \item Mara R. Revkin (2020) What explains taxation by resource-rich rebels? Evidence from the Islamic State in Syria. \textit{Journal of Politics} 82(2).
  \item Robert A. Pape (2003) The Strategic logic of suicide terrorism. \textit{American Political Science Review} 97(3): 343-361.
\end{itemize}

\subsection*{May 5 - Session 11: Student presentations (Iasmin \& Fran)}

\begin{itemize}
  \item No readings
\end{itemize}


\end{document}
