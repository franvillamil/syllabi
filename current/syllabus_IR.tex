\documentclass[12pt, a4paper]{article}
\usepackage[margin = 2cm]{geometry}
\usepackage{graphicx}
\usepackage[english]{babel}
\usepackage[utf8]{inputenc}

\usepackage{color}
	\definecolor{dark_blue}{rgb}{0.2, 0.0, .7}
\usepackage[colorlinks = TRUE,
			allcolors = dark_blue]{hyperref}

\usepackage{setspace}
\setstretch{1.25}
\renewcommand*\rmdefault{ppl}


\usepackage[]{titlesec}
    \titleformat*{\section}{\large\bf}
    \titleformat*{\subsection}{\normalsize\bf}
    \titleformat*{\subsubsection}{\normalsize\it}

%%%%%%%%%%%%%%%%%%%%%%%%%%%%%%%%%%%%
\begin{document}
\begin{center}
{\LARGE\bf International Relations}\\\vspace{10pt}
Master in Social Sciences\\Carlos III-Juan March Institute\\
%\\Universidad Carlos III de Madrid\\
\vspace{10pt}
{\large Spring 2023}\\
\end{center}

\vspace{15pt}

\begin{minipage}{0.6\textwidth}
\textbf{Francisco Villamil}\\
Email: \href{francisco.villamil@uc3m.es}{francisco.villamil@uc3m.es}\\
Office: 18.2.D15 (Thu 15-16h or by appointment)
\end{minipage}\hfill
\begin{minipage}{0.39\textwidth}
\centering
\textbf{Time and place:}\\
Thursdays, 10h--13h\\Room 18.1.A01
\end{minipage}


% \vspace{10pt}
\section{Description}

This is a graduate-level course on contemporary research on international relations. The core of the field focuses on the causes, dynamics, and consequences of aggression and cooperation between states, and has traditionally revolved around the study of war and peace.
We will focus on the study of international conflict, but also on related topics that are often considered within IR, such as the study of internal conflict and processes of political violence, the nature of state size and borders, the influence of the international system on domestic institutions, or processes of international diffusion of norms.
% Although, especially during the first sessions, we will review some classic works that have a strong theoretical component,
We will focus on recent research and on the methodological aspect of studying these topics empirically.

\section{Requirements and assessment}

Each session will consist of a brief lecture introducing the topic and the main concepts, followed by a seminar discussion on the required readings. Students are expected to attend every week, read the required readings in advance, and actively participate in the discussion.
In addition, each student will have to present one of the complementary readings once during the semester.
The last session of the course will be a workshop where students present ideas and ongoing work for the final research paper.

% \begin{itemize}
%   \item[] \textbf{Participation} (30\%):
%   \item[] \textbf{Presentation}: (10\%)
%   \item[] \textbf{Research paper} (60\%):
% \end{itemize}

\paragraph{Participation (30\%):} Students are expected to participation in all seminar discussions. In addition, they have to choose three sessions and prepare a brief memo (around 2 pages) for each of them, summarizing and connecting readings. Submission before class starts.

\vspace{-5pt}\paragraph{Presentation (10\%):} Once during the semester, each student will present one of the complementary readings in class, discussing how it relates and complements the main readings.

\vspace{-5pt}\paragraph{Research paper (60\%):} Students are expected to write a short empirical research paper (maximum 4,000 words) in one of the topics covered in the course, using empirical evidence (qualitative, quantitative, or both) to answer a research question. A proposal will be presented and discussed in the last session in class. Deadline TBD (around late May).

\newpage
\section{Course outline}

 % \hline % ============================================================

\subsection*{Session 1: Introduction}

\noindent\textit{Required:}

\begin{itemize}
  \item Stephen M. Walt (1998) \href{https://doi.org/10.2307/1149275}{International Relations: One World, Many Theories.} \textit{Foreign Policy} 110: 29--32, 34--46.
  \item David A Lake (2011) \href{https://doi.org/10.1111/j.1468-2478.2011.00661.x}{Why `isms' Are Evil: Theory, Epistemology, and Academic Sects as Impediments to Understanding and Progress.} \textit{International Studies Quarterly} 55(2): 465--480.
\end{itemize}

\noindent\textit{Complementary:}

\begin{itemize}
  \item Bruce Bueno de Mesquita (1985) \href{https://doi.org/10.2307/2600500}{Toward a Scientific Understanding of International Conflict: A Personal View.} \textit{International Studies Quarterly} 29(2): 121--136.
\end{itemize}

% - DOMESTIC POLITICS, FOREIGN POLICY, AND THEORIES OF INTERNATIONAL RELATIONS
% - Sikking on constructivism?

\vspace{20pt}
\hline
% ------------------------------------------------------------
\subsection*{Session 2: Interstate war}

\noindent\textit{Required:}

\begin{itemize}
  \item Kenneth N. Waltz (1959) \href{https://cup.columbia.edu/book/man-the-state-and-war/9780231188043}{\textit{Man, the State, and War: A Theoretical Analysis}}. New York: Columbia University Press. [Chapters 2, 4, and 6.]
  \item James D. Fearon (1995) \href{https://doi.org/10.1017/S0020818300033324}{Rationalist Explanations for War.} \textit{International Organization} 49(3): 379--414.
  \item Paul Huth, Christopher Gelpi, and D. Scott Bennett (1993) \href{https://doi.org/10.2307/2938739}{The Escalation of Great Power Militarized Disputes: Testing Rational Deterrence Theory and Structural Realism.} \textit{American Political Science Review} 87(3): 609--623.
\end{itemize}

\noindent\textit{Complementary:}

\begin{itemize}
  \item Kalevi J. Holsti (1991) \href{https://www.cambridge.org/core/books/peace-and-war/37AEB58913E6EF0834D40EFE086D32FE}{\textit{Peace and War: Armed Conflicts and International Order, 1648-1989.}} New York: Cambridge University Press. [esp. Chapter 1, pp. 1--24.]
  % \item Robert Powell (1996) \href{https://doi.org/10.2307/2945840}{Uncertainty, Shifting Power, and Appeasement.} \textit{American Political Science Review} 90(4): 749--764.
  \item James D. Fearon (1994) \href{https://doi.org/10.2307/2944796}{Domestic Political Audiences and the Escalation of International Disputes.} \textit{American Political Science Review} 88(3): 577--592.
	\item Brett Ashley Leeds (2003) \href{https://doi.org/10.1111/1540-5907.00031}{Do Alliances Deter Aggression? The Influence of Military Alliances on the Initiation of Militarized Interstate Disputes}. \textit{American Journal of Political Science} 47(3): 427--439.
\end{itemize}

% \vspace{20pt}
% \hline
% ------------------------------------------------------------
\subsection*{Session 3: State-building, nationalism, and war}

\noindent\textit{Required:}

\begin{itemize}
  \item Charles Tilly (1985) `War making and state making as organized crime.' In \href{https://bibliotecas.uc3m.es/permalink/f/1qk6at5/34UC3M_ALMA21176158990004213}{\textit{Bringing the State Back In}} (ed. P. Evans, D. Rueschemyer \& T. Skocpol), New York: Cambridge University Press, 169--187.
	\item Sascha O. Becker, Andreas Ferrara, Eric Melander, and Luigi Pascali (2020) \href{https://cepr.org/publications/dp15601}{Wars, Taxation and Representation: Evidence from Five Centuries of German History}. \textit{CEPR Discussion Paper} No. 15601.
	\item Andreas Wimmer and Brian Min (2006) \href{https://doi.org/10.1177/000312240607100601}{From Empire to Nation-State: Explaining Wars in the Modern World, 1816–2001}. \textit{American Sociological Review} 71: 867--897.
  \item Agustina S. Paglayan (2022) \href{https://doi.org/10.1017/S0003055422000247}{Education or Indoctrination? The Violent Origins of Public School Systems in an Era of State-Building.} \textit{American Political Science Review} 116(4): 1242--1257.
\end{itemize}

\noindent\textit{Complementary:}

\begin{itemize}
	\item Mark Mazower (2002) \href{https://doi.org/10.1086/ahr/107.4.1158}{Violence and the State in the Twentieth Century}. \textit{American Historical Review} 107(4): 1158--1178.
  \item Michael Mann (2004) \href{https://www.cambridge.org/core/books/dark-side-of-democracy/7E75A132A188A2804E91F4F209B6FE1F}{\textit{The Dark Side of Democracy: Explaining Ethnic Cleansing.}} New York: Cambridge University Press. [Chapter 1, pp. 1--33.]
	\item Lars-Erik Cederman, T.	Camber Warren, and Didier Sornette (2011) \href{https://doi.org/10.1017/S0020818311000245}{Testing Clausewitz: Nationalism, Mass Mobilization, and the Severity of War}. \textit{International Organization} 65(4): 605--638.
  % \item Barry R. Posen (1993) \href{https://doi.org/10.2307/2539098}{Nationalism, the Mass Army, and Military Power}. \textit{International Security} 18(2): 80--124.
\end{itemize}

\vspace{20pt}
\hline
% ------------------------------------------------------------
\subsection*{{\color{red}{Session 4: Democracy, war, and peace}}}

% Oneal, John, Bruce Russett, and Michael Berbaum. 2003. Causes of Peace: Democracy,
% Interdependence, and International Organizations, 1885-1992. International Studies Quarterly 47 (3): 371-394.
% Dixon, William. 1994. Democracy and the Peaceful Settlement of Conflicts. American Political Science Review 88 (1): 14-32.
% Pevehouse, Jon, and Bruce Russett. 2006. Democratic International Governmental Organizations Promote Peace. International Organization 60 (4): 969-1000.
% Layne, Christopher. 1994. Kant or Cant: The Myth of the Democratic Peace. International Security 19 (2): 5-49.
% Mitchell, Sara McLaughlin. 2002. A Kantian System? Democracy and Third Party Conflict Resolution. American Journal of Political Science 46 (4): 749-759.

% War and peace in space and time: The role of democratization Kristian S Gleditsch, Michael D Ward


\noindent\textit{Required:}

\begin{itemize}
  \item
  \item
  \item Cederman, Lars-Erik (2001) \href{https://doi.org/10.1017/S0003055401000028}{Back to Kant: Reinterpreting the Democratic Peace as a Macrohistorical Learning Process.} \textit{American Political Science Review} 95(1): 15--31.
  \item Kristian Skrede Gleditsch and Michael D. Ward (2006) \href{https://doi.org/10.1017/S0020818306060309}{Diffusion and the International Context of Democratization}. \textit{International Organization} 60(4): 911--933.
\end{itemize}

\noindent\textit{Complementary:}

\begin{itemize}
  \item Michael D. Ward and Kristian S. Gleditsch (1998) \href{https://doi.org/10.2307/2585928}{Democratizing for Peace}. \textit{American Political Science Review} 92(1): 51--61.
  \item Michael D. Ward, Randolph M. Siverson, and Xun Cao (2007) \href{https://doi.org/10.1111/j.1540-5907.2007.00269.x}{Disputes, Democracies, and Dependencies: A Reexamination of the Kantian Peace}. \textit{American Journal of Political Science} 51(3): 583--601.
\end{itemize}

\vspace{20pt}
\hline
% ------------------------------------------------------------
\subsection*{{\color{red}{Session 5: Political economy}}}

\noindent\textit{Required:}

\begin{itemize}
  \item Rodrik, Dani (2021) \href{https://www.annualreviews.org/doi/abs/10.1146/annurev-economics-070220-032416}{Why Does Globalization Fuel Populism? Economics, Culture, and the Rise of Right-Wing Populism.} \textit{Annual Review of Economics} 13: 133--170.
  \item
  \item
\end{itemize}

% Dube, Oeindrila and Juan F. Vargas. 2013. "Commodity Price Shocks and Civil Conflict: Evidence from Colombia." The Review of Economic Studies 80(4): 1384–1421.

% Blair, Graeme, Darin Christensen, and Aaron Rudkin. 2021. “Do Commodity Price Shocks Cause Armed Conflict? A Meta-Analysis of Natural Experiments.” American Political Science Review 115(2): 709–16

% Berman, Nicolas, Mathieu Couttenier, Dominic Rohner, and Mathias Thoenig. 2017. "This Mine is Mine! How Minerals Fuel Conflicts in Africa." American Economic Review 107(6): 1564–1610.


\noindent\textit{Complementary:}

\begin{itemize}
  \item
  \item
\end{itemize}

\vspace{20pt}
\hline
% ------------------------------------------------------------
\subsection*{{\color{red}{Session 6: International borders and state size}}}

meter aqui algo de genocide? intl approach

\noindent\textit{Required:}

\begin{itemize}
  \item
  \item
  \item
\end{itemize}

\noindent\textit{Complementary:}

\begin{itemize}
  \item
  \item
\end{itemize}

\vspace{20pt}
\hline
% ------------------------------------------------------------
\subsection*{Session 7: Civil wars}

\noindent\textit{Required:}

\begin{itemize}
  \item James D. Fearon and David Laitin (2003) \href{https://doi.org/10.1017/S0003055403000534}{Ethnicity, Insurgency, and Civil War} \textit{American Political Science Review} 97(1): 75--90.
  \item Lars-Erik Cederman, Andreas Wimmer, and Brian Min (2010) \href{https://doi.org/10.1017/S0043887109990219}{Why Do Ethnic Groups Rebel? New Data and Analysis.} \textit{World Politics} 62(1): 87--119.
  \item Michael Doyle and Nicholas Sambanis (2000) \href{https://doi.org/10.2307/2586208}{International Peacebuilding: A Theoretical and Quantitative Analysis.} \textit{American Political Science Review} 94(4): 779--802.
\end{itemize}

\noindent\textit{Complementary:}

\begin{itemize}
  \item Barry Posen (1993) \href{https://doi.org/10.1080/00396339308442672}{The Security Dilemma and Ethnic Conflict.} \textit{Survival} 35:1: 27-47.
  \item Lars-Erik Cederman, Kristian Skrede Gleditsch, and Nils B. Weidmann (2011) \href{https://doi.org/10.1017/S0003055411000207}{Horizontal Inequalities and Ethnonationalist Civil War: A Global Comparison.} \textit{American Political Science Review} 105(3): 478--495.
  % \item Lars-Erik Cederman and Luc Girardin (2007) \href{https://doi.org/10.1017/S0003055407070086}{Beyond Fractionalization: Mapping Ethnicity onto Nationalist Insurgencies.} \textit{American Political Science Review} 101(1): 173--185.
  \item Michael D. Ward, Nils W. Metternich, Cassy L. Dorff, Max Gallop, Florian M. Hollenbach, Anna Schultz, and Simon Weschle (2013) \href{https://doi.org/10.1111/misr.12072}{Learning from the Past and Stepping into the Future: Toward a New Generation of Conflict Prediction.} \textit{International Studies Review} 15(4): 473--490.
  \item Stathis Kalyvas and Laia Balcells (2010) \href{https://doi.org/10.1017/S0003055410000286}{International System and Technologies of Rebellion: How the End of the Cold War Shaped Internal Conflict}. \textit{American Political Science Review} 104(3): 415--429.
	\item Chaim Kaufmann (1996) \href{https://doi.org/10.1162/isec.20.4.136}{Possible and Impossible Solutions to Ethnic Civil Wars.} \textit{International Security} 20:4: 136-175.
	% \item Nicholas Sambanis (2004) \href{https://doi.org/10.1017/S1537592704040149}{Using Case Studies to Expand Economic Models of Civil War}. \textit{Perspectives on Politics} 2(2):259--279.
\end{itemize}

\vspace{20pt}
\hline
% ------------------------------------------------------------
\subsection*{{\color{red}{Session extra: Disaggregating internal conflict}}}


Networks and conflict behavior
Escalation
Individual participation
etc?
diffs with onset


\begin{itemize}
  \item Güne\c{s} Murat Tezcür (2016) \href{https://doi.org/10.1017/S0003055416000150}{Ordinary People, Extraordinary Risks: Participation in an Ethnic Rebellion}. \textit{American Political Science Review} 110(2): 247--264.
\end{itemize}


\begin{itemize}
  \item Lars-Erik Cederman, Simon Hug, Livia I. Schubiger, and Francisco Villamil (2020) \href{https://doi.org/10.1177/0022002719898873}{Civilian Victimization and Ethnic Civil War}. \textit{Journal of Conflict Resolution} 64(7–8): 1199--1225.
\end{itemize}


\vspace{20pt}
\hline
% ------------------------------------------------------------
\subsection*{{\color{red}{Session 8: Wartime violence}}}

\noindent\textit{Required:}

\begin{itemize}
  \item Stathis Kalyvas (2006) \href{https://bibliotecas.uc3m.es/permalink/f/1nggclj/34UC3M_ALMA21161986050004213}{\textit{The Logic of Violence in Civil War}}. Cambridge University Press, 2006. Chapters 1 (16--31) and 7 (173--209).
  \item Laia Balcells (2010) \href{https://doi.org/10.1111/j.1468-2478.2010.00588.x}{Rivalry and Revenge: Violence against Civilians in Conventional Civil Wars}. \textit{International Studies Quarterly} 54(2): 291--313.
  \item {\color{red}{Weidmann, N. B. (2011). Violence “from above” or “from below”? The Role of Ethnicity in Bosnia’s Civil War. The Journal of Politics, 73(04), 1178-1190.}}
  \item {\color{red}{Cohen, D. K. (2013). Explaining rape during civil war: Cross-national evidence (1980– 2009). American Political Science Review, 107(03), 461-477.}}
\end{itemize}

\noindent\textit{Complementary:}

\begin{itemize}
  \item Robert Braun (2016) \href{https://doi.org/10.1017/S0003055415000544}{Religious Minorities and Resistance to Genocide: The Collective Rescue of Jews in the Netherlands during the Holocaust.} \textit{American Political Science Review} 110(1): 127--147.
  \item Stathis Kalyvas and Matthew A. Kocher (2007) \href{https://doi.org/10.1353/wp.2007.0023}{How `Free' is Free Riding in Civil Wars?: Violence, Insurgency, and the Collective Action Problem}. \textit{World Politics} 59(2): 177--216.
\end{itemize}

\vspace{20pt}
\hline
% ------------------------------------------------------------
\subsection*{Session 9: Terrorism and armed groups}

\noindent\textit{Required:}

% TERRORISM

% Kydd, Andrew. H, and Walter, Barbara. F. 2006. “The Strategies of Terrorism.” International Security 31(1): 49-80.

% David Lake. 2002. “Rational Extremism: Understanding Terrorism in the Twenty First Century.” International Organization 56(1): 15-29.

% Savun, Burcu and Brian J. Phillips. 2009. “Democracy, Foreign Policy, and Terrorism,” Journal of Conflict Resolution 53(6): 878-904.

% Pape, Robert. 2003. “The Strategic Logic of Suicide Terrorism.” American Political Science Review 97(3): 343-361.

% REBEL GOBERNANCE

% Mampilly, Z., & Stewart, M. A. (2021). A Typology of Rebel Political Institutional  Arrangements. Journal of Conflict Resolution, 65(1), 15–45. https://doi.org/10.1177/0022002720935642


\begin{itemize}
  \item
  \item
  \item
\end{itemize}

\noindent\textit{Complementary:}

\begin{itemize}
  \item Nils W. Metternich, Cassy Dorff, Max Gallop, Simon Weschle, and Michael D. Ward (2013) \href{https://doi.org/10.1111/ajps.12039}{Antigovernment Networks in Civil Conflicts: How Network Structures Affect Conflictual Behavior}. \textit{American Journal of Political Science} 57(4): 892--911.
  \item
\end{itemize}

\vspace{20pt}
\hline
% ------------------------------------------------------------
\subsection*{Session 10: Legacies of conflict}

\noindent\textit{Required:}

\begin{itemize}
  \item
  \item
  \item
\end{itemize}

\noindent\textit{Complementary:}

\begin{itemize}
  \item
  \item
\end{itemize}

\vspace{20pt}
\hline
% ------------------------------------------------------------
\subsection*{Session 11: Transitional justice and norm diffusion}

aprovechar y meter rollos de constructivism?

\noindent\textit{Required:}

\begin{itemize}
  \item Kathryn Sikkink and Carrie Booth Walling (2007) \href{https://doi.org/10.1177/002234330707895}{The Impact of Human Rights Trials in Latin America.} \textit{Journal of Peace Research} 44(4): 427--445.
  \item {\color{red}{Kim HJ. 2012. Structural determinants of human rights prosecutions after democratic transition. J. Peace Res. 49(2):305–20}}
  \item {\color{red}{Kim HJ, Sikkink K. 2010. Explaining the deterrence effect of human rights prosecutions for transitional countries. Int. Stud. Q. 54:939–63}}
\end{itemize}

Wendt, Alexander. 1992. Anarchy Is What States Make Of It: The Social Construction of Power
Politics. International Organization 46 (4): 887-918.
Finnemore, Martha, and Kathryn Sikkink. 1998. International Norm Dynamics and Political
Change. International Organization 52 (4): 887-918.
Adler, Emmanuel. 1997. “Seizing the Middle Ground: Constructivism in World Politics.”
European Journal of International Relations 3, 3: 319-63
Ruggie, John Gerard. 1986. Continuity and Transformation in the World Polity: Toward a
Neorealist Synthesis. In Robert Keohane, Ed. Neorealism and Its Critics. New York: Columbia University Press: 131-157.
Katzenstein, Peter. 2003. Same War-Different Views: Germany, Japan, and Counterterrorism.
International Organization 57 (4): 731-760.
Lebovic, James, and Erik Voeten. 2006. The Politics of Shame: The Condemnation of Country
Human Rights Practices in the UNHCR. International Studies Quarterly 50 (4): 861-888.


\noindent\textit{Complementary:}

\begin{itemize}
  \item
  \item
\end{itemize}

\hline
% ------------------------------------------------------------
\subsection*{Session 12: Research workshop}

No readings. Presentation of research projects and workshop discussion.

\end{document}
