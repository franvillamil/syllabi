% NOTE por 2023: igual mejor hacer presentations + essay by groups

\documentclass[12pt, a4paper]{article}
\usepackage[margin = 2cm]{geometry}
\usepackage{graphicx}
\usepackage[english]{babel}
\usepackage[utf8]{inputenc}
\usepackage[colorlinks = TRUE]{hyperref}
\usepackage{setspace}
\setstretch{1.25}
\renewcommand*\rmdefault{ppl}


\usepackage[]{titlesec}
    \titleformat*{\section}{\large\bf}
    \titleformat*{\subsection}{\normalsize\bf}
    \titleformat*{\subsubsection}{\normalsize\it}

%%%%%%%%%%%%%%%%%%%%%%%%%%%%%%%%%%%%
\begin{document}
\begin{center}
{\LARGE\bf Research Design}\\\vspace{10pt}
Master in Computational Social Science\\
Universidad Carlos III de Madrid\\\vspace{10pt}
{\large Fall 2022}\\
\end{center}

\vspace{20pt}

\begin{minipage}{0.49\textwidth}
\centering
Francisco Villamil\\
\href{francisco.villamil@uc3m.es}{francisco.villamil@uc3m.es}\\
Office: 18.2.D15 (Getafe)\\
Office hours by appointment
\end{minipage}\hfill
\begin{minipage}{0.49\textwidth}
\centering
\textbf{Tuesday 15:00--18:00h}\\Room 2.A.04 (Puerta de Toledo)\\September 13th - October 25th\\
\end{minipage}


\vspace{10pt}
\section{Description}

This course provides an introduction to research design in the social sciences.
The goal is to equip students with the skills necessary to evaluate and develop strategies to answer empirical questions with data.
We will not cover statistical techniques or advanced data analysis.
Rather, our focus will be on the logic of empirical comparison, particularly applied to quantitative data.

\section{Requirements}

We meet once a week for a total of seven sessions. During the first six weeks, we will review the main topics in research design, including the basics of empirical evidence, the link between theory and empirics, or the logic of causal inference. Every session will include a lecture followed by practical exercises related to what was covered that day. Some of these exercises might require previous readings, which will be made available the previous week. The final session will consist of a workshop where students both present an ongoing project where they develop an strategy to answer empirically to a research question and comment on each other's projects. Attendance to all sessions is mandatory.

\section{Materials}

This course does not follow any textbook in particular.
However, there are some books that were used to design the course and can be very useful to expand on what is covered in class:

\begin{itemize}
\setlength\itemsep{-5pt}
  \item Nick Huntington-Klein, \textit{The Effect: An Introduction to Research Design and Causality} (Chapman and Hall/CRC Publishing, 2021).
  \item Ethan Bueno de Mesquita \& Anthony Fowler, \textit{Thinking clearly with data: A guide to quantitative reasoning and analysis} (Princeton University Press, 2021).
  \item Scott Cunningham, \textit{Causal Inference: The Mixtape} (Yale University Press, 2021).
  \item Paul M. Kellstedt \& Guy D. Whitten, \textit{The Fundamentals of Political Science Research} (Cambridge University Press, 2018).
  % \item Philip H. Pollock, \textit{The Essentials of Political Analysis} (Sage, 2016)
  % \item Matthew Salganik ?
\end{itemize}

Some of these books (Huntington-Klein's \textit{The Effect} and Cunningham's \textit{Mixtape}) are available online for free.\footnote{\url{https://theeffectbook.net/} and \url{https://mixtape.scunning.com/}.} Besides, I will make available additional readings in \textit{Aula Global}.

\section{Assessment}

% Students will be evaluated based on three different assignments:

\subsection*{Participation (10\%)}

Every student is expected to be an active participant in all sessions, asking questions and engaging in discussions, including during the lectures.

\subsection*{In-class exercises (30\%)}

In every session, during the second half, we will make practical exercises related to what was covered each day in class. These exercises can be made individually or in groups.

\subsection*{Workshop presentation (20\%)}

The last session will be a workshop where students present an ongoing project, corresponding to their final essay. This project should present a research question that can be answered empirically with quantitative data and a strategy to answer it. Rather than focusing on data analysis, the focus should be on the type of variation that will be exploited and how well it answers the question. A potential structure could be:

\begin{itemize}
\setlength\itemsep{-5pt}
  \item Present a topic in general terms, why is it relevant?
  \item Specific research question
  \item Empirical strategy:
  \vspace{-10pt}
  \begin{itemize}
  \setlength\itemsep{-5pt}
    \item What type of variation is going to be exploited? Unit of analyses
    \item Data: source, variables, measurement
    \item Empirical comparisons
  \end{itemize}
\end{itemize}

Student will also comment on each other's project, looking for limitations and possible ways forward. Evaluation will focus more on the presentation and the Q\&A than on the content itself.

\subsection*{Final essay (40\%)}

The main assignment is a written essay developing the research design. This assignment can be thought of as a pre-analysis plan for some study, but there are other options (e.g. compare two different strategies for the same research question).

\newpage
\section{Course outline}

% \hline % ============================================================
\subsection*{September 13th: Introduction to social research}

\begin{itemize}
\setlength\itemsep{-5pt}
  \item What is social \textit{science}?
  \item Using empirical evidence to answer questions
  \item Importance of research design
  \item Types of research questions and types of empirical research
\end{itemize}

\subsection*{September 20th: Social mechanisms and causality}

\begin{itemize}
\setlength\itemsep{-5pt}
  \item What are we explaining? Building blocks: units, variables, processes
  \item Simplifying social complexity
  \item Different units of analysis and mechanisms
  \item How to approach this empirically? Key idea: variability
\end{itemize}

\subsection*{September 27th: Understanding empirical evidence}

\begin{itemize}
\setlength\itemsep{-5pt}
  \item What is \textit{data}? Types of empirical evidence
  \item Unit of analysis and variability, different approaches %Case studies, comparative studies, quantitative studies with large samples
  \item Raw data and final data, aggregating and disaggregating, measurement problems
\end{itemize}

\subsection*{October 4th: Causal inference design}

\begin{itemize}
\setlength\itemsep{-5pt}
  \item The experimental ideal in the natural and social sciences
  \item How to get closer with observational data?
  \item Confounding, selection bias, collider bias, etc
\end{itemize}

\subsection*{October 11th: Assessing external and internal validity}

\begin{itemize}
\setlength\itemsep{-5pt}
  \item Bringing it all together
  \item Where does evidence come from? External validity
  \item Issues with unit of analysis? Ecological fallacy
  \item Unaccounted for processes? Diffusion, unit independence
\end{itemize}

\subsection*{October 18th: Introduction to causal inference methods}

\begin{itemize}
\setlength\itemsep{-5pt}
  \item Understanding the logic of methods to determine causality, without the use of statistics
  \item Controls, matching
  \item Exploiting exogeneity: difference-in-differences, regression discontuinity
\end{itemize}

\noindent
\textbf{(*)} Depending on timing, there will be a final session on \textbf{quantitative research workflow}.

\subsection*{October 25th: Presentations workshop}

\subsection*{$*$ Final essay {\color{red}{deadline}}: November 4th, 23.59h}

\end{document}
