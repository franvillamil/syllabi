% NOTE: next year nada de discussion leaders, no funciona bien.
% probablemente mejor sea hacer una presentación todos los días y un debate posterior.
% quizás focusing on case studies? hay unas 14-15 semanas, son grupos de 2-3 personas

% e.g: in all the seminars, choose a reading about one example (many already are)
% and make one group present the case and the interpretation in the reading
% - followed by debate

% Also NOTE: LAPTOP POLICY FOR SEMINARS?


\documentclass[12pt, a4paper]{article}
\usepackage[margin = 2cm]{geometry}
\usepackage{graphicx}
\usepackage[english]{babel}
\usepackage[utf8]{inputenc}
\usepackage[colorlinks = TRUE]{hyperref}
\usepackage{setspace}
\setstretch{1.25}
\renewcommand*\rmdefault{ppl}


\usepackage[]{titlesec}
    \titleformat*{\section}{\large\bf}
    \titleformat*{\subsection}{\normalsize\bf}
    \titleformat*{\subsubsection}{\normalsize\it}

%%%%%%%%%%%%%%%%%%%%%%%%%%%%%%%%%%%%
\begin{document}
\begin{center}
{\LARGE War, peace, and political violence}\\\vspace{10pt}
BA History and Politics\\
Universidad Carlos III de Madrid\\\vspace{10pt}
{\large Fall 2022}\\
\end{center}

\vspace{20pt}

\begin{minipage}{0.49\textwidth}
\centering
Francisco Villamil\\
\href{francisco.villamil@uc3m.es}{francisco.villamil@uc3m.es}\\
Office: 18.2.D15\\
Office hours: by appointment
\end{minipage}\hfill
\begin{minipage}{0.49\textwidth}
\centering
\textbf{Lecture} (17.1.04)\\Monday 14:30--16:00h\\\vspace{5pt}
\textbf{Seminar} (14.1.01)\\Thursday 10:45--12:15h\\
\end{minipage}


\vspace{10pt}
\section{Description}

This course provides an overview over a wide range of topics in conflict research, including inter-state wars, civil wars, and causes and dynamics of political violence.
Its main goal is to provide students with the conceptual and theoretical tools to think analytically about conflicts and political violence. Some of the questions we will explore are: Why do countries fight each other? How do changes in the international system impact conflicts across the world? What explains the outbreak of civil wars? Why and how civilians are killed during wars? What are the long-term consequences of conflicts?

\section{Requirements}

We meet twice a week. In the lectures, we will review the main debates in each topic. Each lecture has one reading assigned, usually a research article, that covers part or most of what we will talk about. Reading it is not mandatory, but recommended, either before or after the lecture. In each seminar, we will discuss a reading related to the lecture. These readings, which are \textbf{mandatory}, are shorter and lighter than the ones assigned to the lectures, and are meant to reflect or expand on the topic covered each week. Each week, some students (depending in class size) will lead the discussion, which counts for the final course grade.

\section{Materials}

You can find all the reading materials in \textit{Aula Global}.

\newpage
\section{Assessment}

% Students will be evaluated based on three different assignments:

\subsection*{Participation (10\%)}

Everyone is expected to attend all sessions and be an active participant in the discussions, especially in the seminar sessions.

\subsection*{Discussion leader (15\%)}

During every seminar, some students will act as discussion leaders. This means that they need to closely read the text before the class and prepare some talking points to discuss in class, ideally relating the seminar reading with what was seen in the lecture. \textbf{Bring them printed to class or send them to me by email} before the end of the class.

There is a \textbf{limit of how many students} act as discussion leaders \textbf{each day}, so not everyone chooses the same day. You can choose the day in the following Doodle poll (first come first served): \href{https://doodle.com/meeting/participate/id/aO7wNypa}{https://doodle.com/meeting/participate/id/aO7wNypa}

% \begin{itemize}
% \setlength\itemsep{0pt}
% \item[] \href{https://doodle.com/meeting/participate/id/aO7wNypa}{https://doodle.com/meeting/participate/id/aO7wNypa}
% \end{itemize}

\subsection*{Presentation (15\%)}

In the last two seminar days, students will have to give a 10-15min group presentation, which will be followed by a short Q\&A. Grading will be based on both the presentation and participation during the Q\&A.

We can discuss in class alternatives but, in principle, two options are: a) an overview of a single conflict, reflecting on one or more topics covered in class (for example: `Violence against civilians in Syria'), or b) a topic, expanding what we covered in class (for example: `Nationalism and conflict in the 21st Century').

\subsection*{Final exam (60\%) -- January}

Two options for the final exam:

\begin{itemize}
  \item[1.] A final take-home exam. Questions will be handed out at least 24h before deadline. Its goal is to evaluate how well students understood the main concepts and ideas.
  \item[2.] A book review of a relevant book, commenting some of the topics discussed in class \textbf{(max 2,500 words)}. Some of the pre-approved options are:
  {\small
  \item[-] P. Radden Keefe, \textit{Say Nothing: A True Story of Murder and Memory in Northern Ireland} \vspace{-10pt}
  \item[-] A. Gopal, \textit{No Good Men Among the Living: America, the Taliban, and the War Through Afghan Eyes} \vspace{-10pt}
  \item[-] W. Finnegan, \textit{A Complicated War: The Harrowing of Mozambique} \vspace{-10pt}
  \item[-] S. Subramanian, \textit{This Divided Island: Life, Death, and the Sri Lankan War} \vspace{-10pt}
  \item[-] V. Bevins, \textit{The Jakarta Method: Washington's Anticommunist Crusade and the Mass Murder Program that Shaped Our World}
  }
  \item[] (Any other option is possible, but needs to be \textbf{previously approved})
\end{itemize}

\newpage
\section{Course outline}

\subsection*{Week 1: Introduction}

\subsubsection*{Sept 5 - Lecture}

Presentation. Course structure and organizational issues. Introduction: what is political violence and what are we going to talk about?

\begin{itemize}
\setlength\itemsep{0pt}
\item Stathis Kalyvas, `The landscape of political violence.' \textit{The Oxford Handbook of Terrorism}, chapter 2, 2019.
\end{itemize}

\subsubsection*{Sept 8 - Seminar}

Discussion of main concepts.

\begin{itemize}
\setlength\itemsep{0pt}
\item Stephen Buckley, The Termite Coup. \textit{The Atlantic,} 06/01/2022.
\item[] \href{https://www.theatlantic.com/ideas/archive/2022/01/january-6-coup-never-ending/621171/}{(theatlantic.com/ideas/archive/2022/01/january-6-coup-never-ending/621171/)}
\end{itemize}

\hline % ============================================================

\subsection*{Weeks 2 \& 3: Classic IR views on wars}

Inter-state wars and classical explanations of warfare. The three visions in IR. Decent decline of inter-state war, the democratic peace and economic interdependences.

\subsubsection*{Sept 12 - Lecture}

\begin{itemize}
\setlength\itemsep{0pt}
\item Chapter 1 in Kalevi J Holsti, \textit{Peace and War: Armed Conflicts and International Order, 1648-1989.} Cambridge UP, 1991: pp. 1--24.
\end{itemize}

\subsubsection*{Sept 15 {\color{red}{\textbf{(NO CLASS)}}}}

\subsubsection*{Sept 19 {\color{red}{\textbf{(NO CLASS)}}}}

\subsubsection*{Sept 22 - Seminar}

\begin{itemize}
\setlength\itemsep{0pt}
\item Kathrin Hille \& Demetri Sevastopulo, `Taiwan: preparing for a potential Chinese invasion.' \textit{Finantial Times,} 07/06/2022. \href{https://www.ft.com/content/0850eb67-1700-47c0-9dbf-3395b4e905fd}{(ft.com/content/0850eb67-1700-47c0-9dbf-3395b4e905fd)}
\end{itemize}

\hline % ============================================================

\subsection*{Week 4: State-building}

State-building and war: origins of the state, role of international conflict in the creation of states and vice-versa, role of violence in the internal history of states. The new international order after 1648.

\subsubsection*{Sept 26 - Lecture}

\begin{itemize}
\setlength\itemsep{0pt}
\item Chapter 1 in Charles Tilly, \textit{Coercion, capital, and European states, AD 990-1992.} Blackwell Publishing, 1990: pp. 1--37.
\end{itemize}

\subsubsection*{Sept 29 - Seminar}

\begin{itemize}
\setlength\itemsep{0pt}
\item John Reed, Guy Chazan \& Roman Olearchyk, `The birth of a new Ukraine: how Russia's war united a nation' \textit{Finantial Times,} 17/03/2022. \href{https://www.ft.com/content/9ab50dee-67f5-4e1b-8456-d8f11814ef18}{(ft.com/content/9ab50dee-67f5-4e1b-8456-d8f11814ef18)}
\end{itemize}


\hline % ============================================================

\subsection*{Week 5: Nation-states}

The French Revolution not only changed peaceful politics, but also war. The development of nationalisms and its relationship with political violence. 'The people' as a source of political legitimacy, wars of independencia and civil wars. Recruitment and mobilization capacity of the new Nation-state.

\subsubsection*{Oct 3 - Lecture}

\begin{itemize}
\setlength\itemsep{0pt}
\item Andreas Wimmer and Brian Min, `From empire to nation-state. Explaining wars in the modern world, 1816-2001.' \textit{American Sociological Review} 71: 867--897, 2006.
\end{itemize}

\subsubsection*{Oct 6 - Seminar}

\begin{itemize}
\setlength\itemsep{0pt}
\item Jill Lepore, `When constitutions took over the world.' \textit{The New Yorker,} 22/03/2021. \href{https://www.newyorker.com/magazine/2021/03/29/when-constitutions-took-over-the-world}{(newyorker.com/magazine/2021/03/29/when-constitutions-took-over-the-world)}
\end{itemize}

\hline % ============================================================

\subsection*{Weeks 6 \& 7: Civil wars}

Basic concepts and types of civil wars. After 1990, there is a deep increase in the outbreak of civil wars. What used to be explained as popular revolutions, now is seen as a problem of anarchy, looting, and greed. Modern grievance-based explanations highlight the role of political inequality (especially along ethnic lines) in increasing the risk of war onset. A new consensus includes both motivation and opportunity factors.

\subsubsection*{Oct 10 - Lecture}

\begin{itemize}
\setlength\itemsep{0pt}
\item James Fearon \& David Laitin, `Ethnicity, insurgency, and civil war.' \textit{American Political Science Review} 97(1): 75--90, 2003.
\end{itemize}

\subsubsection*{Oct 13 - Seminar}

\begin{itemize}
\setlength\itemsep{-5pt}
\item Robert D Kaplan, `The Coming Anarchy: How scarcity, crime, overpopulation, tribalism, and disease are rapidly destroying the social fabric of our planet.' \textit{The Atlantic,} February 1994.
\item[] \href{https://www.theatlantic.com/magazine/archive/1994/02/the-coming-anarchy/304670/}{(theatlantic.com/magazine/archive/1994/02/the-coming-anarchy/304670/)}
\end{itemize}


\subsubsection*{Oct 17 - Lecture}

\begin{itemize}
\setlength\itemsep{0pt}
\item Chps 1 \& 2 in Lars-Erik Cederman, Kristian Skrede Gleditsch \& Halvard Buhaug, \textit{Inequality, grievances, and civil war.} Cambridge UP, 2013: pp. 1--29.
\end{itemize}

\subsubsection*{Oct 20 - Seminar}

\begin{itemize}
\setlength\itemsep{0pt}
\item Anand Gopal, `The other Afghan women' \textit{The New Yorker,} 06/09/2021.
\item[] \href{https://www.newyorker.com/magazine/2021/09/13/the-other-afghan-women}{(newyorker.com/magazine/2021/09/13/the-other-afghan-women)}
\end{itemize}

\hline % ============================================================

\subsection*{Week 8 \& 9: Wartime violence}

The repertoire of violence during wars. Types of violence and definitions. Focus on violence against civilians. Causes and dynamics. Ethnic violence and genocide.

\subsubsection*{Oct 24 - Lecture}

\begin{itemize}
\setlength\itemsep{0pt}
\item Benjamin Valentino, `Why we kill: The political science of political violence against civilians.' \textit{Annual Review of Political Science} 17: 89--103, 2014.
\end{itemize}


\subsubsection*{Oct 27 - Seminar}

\begin{itemize}
\setlength\itemsep{0pt}
\item Patrick Radden Keefe, \textit{Say Nothing: A True Story of Murder and Memory in Northern Ireland. }Harper Collins, 2018: chapters 5--8.
% \item Chapter 1 in Beatriz Manz, \textit{Paradise in ashes: A Guatemalan journey of courage, terror, and hope.} U of California Press, 2004: pp. 1--32.
\end{itemize}

\subsubsection*{Oct 31 {\color{red}{\textbf{(NO CLASS)}}}}

\subsubsection*{Nov 3 - Seminar}

\begin{itemize}
\setlength\itemsep{0pt}
\item William Finnegan, \textit{A complicated war: The harrowing of Mozambique.} U of California Press, 1992: chapters 9 \& 10.
\end{itemize}

\hline % ============================================================

\subsection*{Week 10: Rebel groups and wartime governance}

What happens behind the fronts? Rebel governance and recruitment. How do armed groups control the civilian population? Wartime social processes.

\subsubsection*{Nov 7 - Lecture}

\begin{itemize}
\setlength\itemsep{0pt}
\item Chapter 1 (`Introduction') in Ana Arjona, Nelson Kasfir \& Zachariah Mampilly, \textit{Rebel governance in civil war.} Cambridge UP, 2015: pp. 1--20.
\end{itemize}

\subsubsection*{Nov 10 - Seminar}

\begin{itemize}
\setlength\itemsep{-5pt}
\item Joshua Yaffa, `A Ukrainian city under a violent new regime.' \textit{The New Yorker}, 16/05/2022. \href{https://www.newyorker.com/magazine/2022/05/23/a-ukrainian-city-under-a-violent-new-regime}{(newyorker.com/magazine/2022/05/23/a-ukrainian-city-under-a-violent-new-regime)}
\end{itemize}


\hline % ============================================================

\subsection*{Week 11: Terrorism}

Despite its relevance, terrorism is usually misunderstood. Terrorist actions and terrorist groups. Dynamics and causes. Suicide terrorism.

\subsubsection*{Nov 14 - Lecture}

\begin{itemize}
\setlength\itemsep{0pt}
\item Luis de la Calle \& Ignacio Sánchez-Cuenca, `What we talk about when we talk about terrorism.' \textit{Politics \& Society} 39(3): 451--472, 2011.
\end{itemize}

\subsubsection*{Nov 17 - Seminar}

\begin{itemize}
\setlength\itemsep{0pt}
\item David Pilling, `Niger: the west's bulwark against jihadis and Russian influence in Africa.' \textit{Finantial Times,} 07/07/2022. \href{https://www.ft.com/content/744bea94-3b18-47d5-8e53-e87ab9efef9a}{(ft.com/content/744bea94-3b18-47d5-8e53-e87ab9efef9a)}
\end{itemize}

\hline % ============================================================

\subsection*{Week 12: Postwar politics}

Strategies to deal with conflict-ridden contries. Effects of power-sharing, regional autonomy, and secession. Postwar democratization. Combatatant demobilization. Transitional justice.

\subsubsection*{Nov 21 - Lecture}

\begin{itemize}
\setlength\itemsep{0pt}
\item Hanna Leonardsson \& Gustav Rudd, `The ‘local turn’ in peacebuilding: a literature review of effective and emancipatory local peacebuilding.' \textit{Third World Quarterly} 36(5): 825-839, 2015.
\end{itemize}

\subsubsection*{Nov 24 - Seminar}

\begin{itemize}
\setlength\itemsep{0pt}
\item Jon Lee Anderson, `The Taliban confront the realities of power' \textit{The New Yorker,} 21/02/2022. \href{https://www.newyorker.com/magazine/2022/02/28/the-taliban-confront-the-realities-of-power-afghanistan}{(newyorker.com/magazine/2022/02/28/the-taliban-confront-the-realities-of-power-afghanistan)}
\end{itemize}


\hline % ============================================================

\subsection*{Week 13: Long-term legacies and trends}

Wars, especially wartime violence, transform countries and societies fundamentally. Consequences of civil wars on the civilian population. Long-term legacies on preferences.

\subsubsection*{Nov 28 - Lecture}

\begin{itemize}
\setlength\itemsep{0pt}
\item Jacob Walden and Yuri M. Zhukov, `Historical legacies of political violence.' In \textit{Oxford Research Encyclopedia of Politics}, 2020.
\end{itemize}

\subsubsection*{Dec 1 - Seminar}

\begin{itemize}
\setlength\itemsep{0pt}
\item Benjamin Wallace-Wells, `The Fight Over Virginia’s Confederate Monuments' \textit{The New Yorker,} 27/11/2017. \href{https://www.newyorker.com/magazine/2017/12/04/the-fight-over-virginias-confederate-monuments}{(newyorker.com/magazine/2017/12/04/the-fight-over-virginias-confederate-monuments)}
\end{itemize}

% Masha Gessen, `The Prosecution of Russian War Crimes in Ukraine' \textit{The New Yorker,} 01/08/2022. \href{https://www.newyorker.com/magazine/2022/08/08/the-prosecution-of-russian-war-crimes-in-ukraine}{(newyorker.com/magazine/2022/08/08/the-prosecution-of-russian-war-crimes-in-ukraine)}


\hline % ============================================================

\subsection*{Weeks 14 \& 15: Student presentations}

\subsubsection*{Dec 5 - Presentations}

\subsubsection*{Dec 8 {\color{red}{\textbf{(NO CLASS)}}}}

\subsubsection*{Dec 12 - Presentations}

\end{document}
