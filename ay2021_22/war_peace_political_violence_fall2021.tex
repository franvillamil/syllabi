\documentclass[12pt, a4paper]{article}
\usepackage[margin = 2cm]{geometry}
\usepackage{graphicx}
\usepackage[english]{babel}
\usepackage[utf8]{inputenc}
\usepackage[colorlinks = TRUE]{hyperref}
\usepackage{setspace}
\setstretch{1.25}
\renewcommand*\rmdefault{ppl}


\usepackage[]{titlesec}
    \titleformat*{\section}{\large\bf}
    \titleformat*{\subsection}{\normalsize\bf}
    \titleformat*{\subsubsection}{\normalsize\it}

%%%%%%%%%%%%%%%%%%%%%%%%%%%%%%%%%%%%
\begin{document}
\begin{center}
{\Large War, peace, and political violence}\\\vspace{10pt}
Bachelor in History and Politics\\
Universidad Carlos III de Madrid\\
Fall 2021\\
\end{center}

\vspace{20pt}

\begin{minipage}{0.49\textwidth}
\centering
Francisco Villamil\\
\href{francisco.villamil@uc3m.es}{francisco.villamil@uc3m.es}\\
Office: 18.2.E05\\
Office hours: by appointment
\end{minipage}\hfill
\begin{minipage}{0.49\textwidth}
\centering
Lecture (17.1.04)\\Monday 14:30--16:00h\\\vspace{5pt}
Seminar (14.1.01)\\Thursday 10:45--12:15h\\
\end{minipage}


\vspace{10pt}
\section{Description}

This course provides an overview over a wide range of topics in conflict research, including inter-state wars, civil wars, and causes and dynamics of political violence.
Its main goal is to provide students with the conceptual and theoretical tools to think analytically about conflicts and political violence. Some of the questions we will explore are: Why do countries fight each other? How do changes in the international system impact conflicts across the world? What explains the outbreak of civil wars? Why and how civilians are killed during wars? What are the long-term consequences of conflicts?

\section{Requirements}

We meet twice a week. In the lectures, we will review the main debates in each topic. Each lecture has one reading assigned, usually a research article, that covers part or most of what we will talk about. Reading it is not mandatory, but recommended, either before or after the lecture. In each seminar, we will discuss a reading related to the lecture. These readings, which are \textbf{mandatory}, are shorter and lighter than the ones assigned to the lectures, and are meant to reflect or expand on the topic covered each week.

\section{Materials}

You can find all the reading materials in \textit{Aula Global}.

\newpage
\section{Assessment}

Students will be evaluated based on three different assignments:

\subsection*{Response papers and participation (20\%)}

Each student will have to choose \textbf{two} seminar dates and write a short comment (around 1 page, max 2) of the readings. Response papers will be due on the seminar day chosen, and participation in seminar will also count towards the grade. Dates can be selected in the Doodle link below:

\begin{itemize}
\setlength\itemsep{0pt}
\item[] \href{https://doodle.com/poll/wf3q29s9tzd2expy}{https://doodle.com/poll/wf3q29s9tzd2expy}
\end{itemize}

\subsection*{Presentation (20\%)}

In the last two or three seminar days (depending on class size), in groups of 2-3 people, students will have to give a 15-min presentation, which will be followed by a short Q\&A. Grading will be based on both the presentation and participation during the Q\&A. We can discuss in class alternatives but, in principle, two options are:

\begin{itemize}
\setlength\itemsep{0pt}
\item[a)] An overview of a single conflict, reflecting on one or more topics covered in class (for example: `Violence against civilians in Syria')
\item[b)] A topic and its relevance in modern or historical times, expanding what we covered in class (for example: `Nationalism and conflict in the 21st Century')
\end{itemize}

\subsection*{Final exam (60\%) -- January 13th, 2020}

Two options for the final exam:

\begin{itemize}
  \item[1.] A final take-home exam. Questions will be handed out at least 24h before deadline. Its goal is to evaluate how well students understood the main concepts and ideas.
  \item[2.] A book review of one of the following books, commenting some of the topics discussed in class (max 2,500 words). The options are:
  \begin{itemize}
  \setlength\itemsep{0pt}
  {\small
    \item Patrick Radden Keefe, \textit{Say Nothing: A True Story of Murder and Memory in Northern Ireland} (William Collins, 2018)
    \item Anand Gopal, \textit{No Good Men Among the Living: America, the Taliban, and the War Through Afghan Eyes} (Picador, 2014)
    \item William Finnegan, \textit{A Complicated War: The Harrowing of Mozambique} (UC Press, 1992)
    \item Samanth Subramanian, \textit{This Divided Island: Life, Death, and the Sri Lankan War} (Penguin, 2015)
    \item Beatriz Manz, \textit{Paradise in Ashes: A Guatemalan Journey of Courage, Terror, and Hope} (UC Press, 2004)
  }
  \end{itemize}
\end{itemize}

\newpage
\section{Course outline}

\subsection*{Sept 6 - Lecture: Introduction}

Presentation. Course structure and organizational issues. Introduction: what is political violence and what are we going to talk about?

\begin{itemize}
\setlength\itemsep{-5pt}
\item \textit{No readings}
\end{itemize}

\subsection*{Sept 9 - Seminar: {\color{red}{No class}}}

% ============================================================
\subsection*{Sept 13 - Lecture: Concepts}

The typologies of conflict and violence. Why violence is not enough to identify a war. The problem of defining the start and end of conflicts. Basic ideas: actors involved, types of violence, objectives, methods, non-fatal violence and repression.

\begin{itemize}
\setlength\itemsep{0pt}
\item Stathis Kalyvas, `The ontology of political violence.' \textit{Perspectives on Politics} 1(3): 475--494, 2003.
\end{itemize}

\subsection*{Sept 16 - Seminar:}

\begin{itemize}
\setlength\itemsep{-5pt}
\item Luke Mogelson, `The militias against masks.' \textit{The New Yorker}, 24/08/2020.
\item[] \href{https://www.newyorker.com/magazine/2020/08/24/the-militias-against-masks}{(www.newyorker.com/magazine/2020/08/24/the-militias-against-masks)}
\end{itemize}

% ============================================================
\subsection*{Sept 20 (Lecture) {\color{red}{No class}}}

\subsection*{Sept 23 - Seminar:}

\begin{itemize}
\setlength\itemsep{-5pt}
\item George Packer, `War after the war: What Washington doesn't see in Iraq.' \textit{The New Yorker}, 17/11/2003.
\item[] \href{https://www.newyorker.com/magazine/2003/11/24/war-after-the-war}{(www.newyorker.com/magazine/2003/11/24/war-after-the-war)}
\end{itemize}

% ============================================================
\subsection*{Sept 27 - Lecture: Classic views of inter-state wars}

Inter-state wars and classical explanations of warfare. The three visions in IR. Decent decline of inter-state war, the democratic peace and economic interdependences.

\begin{itemize}
\setlength\itemsep{0pt}
\item Chapter 1 in Kalevi J Holsti, \textit{Peace and War: Armed Conflicts and International Order, 1648-1989.} Cambridge UP, 1991: pp. 1--24.
\end{itemize}

\subsection*{Sept 30 - Seminar:}

\begin{itemize}
\setlength\itemsep{0pt}
\item Graham Allison, `The Thucydides' Trap: Are the U.S. and China headed for war?' \textit{The Atlantic,} 24/09/2015.
\item[] \href{https://www.theatlantic.com/international/archive/2015/09/united-states-china-war-thucydides-trap/406756/}{www.theatlantic.com/international/archive/2015/09/united-states-china-war-thucydides-trap/406756/}
\end{itemize}

% ============================================================
\subsection*{Oct 4 - Lecture: State-building, nationalism, and inter-state wars}

State-building and war: origins of the state, role of international conflict in the creation of states and vice-versa, role of violence in the internal history of states. The new international order after 1648. The development of nationalisms after the French Revolution and its relationship with political violence. 'The people' as a source of political legitimacy, wars of independencia and civil wars. Recruitment and mobilization capacity of the new Nation-state.

\begin{itemize}
\setlength\itemsep{0pt}
\item Chapter 1 in Charles Tilly, \textit{Coercion, capital, and European states, AD 990-1992.} Blackwell Publishing, 1990: pp. 1--37.
\item Lars-Erik Cederman, T. Camber Warren \& Didier Sornette, `Testing Clausewitz: Nationalism, Mass Mobilization, and the Severity of War.' \textit{International Organization} 65(4): 605--638, 2011.
\end{itemize}

\subsection*{Oct 7 - Seminar:}

\begin{itemize}
\setlength\itemsep{0pt}
\item Chp 1 in Michael Mann, \textit{The dark side of democracy: Explaining ethnic cleansing.} Cambridge UP, 2004: 1--33.
\end{itemize}

% ============================================================
\subsection*{Oct 11 - Lecture: {\color{red}{No class}}}

\subsection*{Oct 14 - Seminar:}

\begin{itemize}
\setlength\itemsep{0pt}
\item Chp 5 in James C Scott, \textit{Against the Grain: A Deep History of the Earliest States.} Yale UP, 2017: pp. 150--182.
\end{itemize}

% ============================================================
\subsection*{Oct 18 - Lecture: Civil wars and greed-based explanations}

Basic concepts and types of civil wars. After 1990, there is a deep increase in the outbreak of civil wars. What used to be explained as popular revolutions, now is seen as a problem of anarchy, looting, and greed.

\begin{itemize}
\setlength\itemsep{0pt}
\item James Fearon \& David Laitin, `Ethnicity, insurgency, and civil war.' \textit{American Political Science Review} 97(1): 75--90, 2003.
\end{itemize}

\subsection*{Oct 21 - Seminar:}

\begin{itemize}
\setlength\itemsep{-5pt}
\item Robert D Kaplan, `The Coming Anarchy: How scarcity, crime, overpopulation, tribalism, and disease are rapidly destroying the social fabric of our planet.' \textit{The Atlantic,} February 1994.
\item[] \href{https://www.theatlantic.com/magazine/archive/1994/02/the-coming-anarchy/304670/}{www.theatlantic.com/magazine/archive/1994/02/the-coming-anarchy/304670/}
\end{itemize}

% ============================================================
\subsection*{Oct 25 - Lecture: Civil wars and grievance-based explanations}

Before 1990, civil wars were usually explained as popular revolutions caused by resentment and inequality. This explanation was sidelined after the `New Wars' of the 1990s. Modern grievance-based explanations highlight the role of political inequality (especially along ethnic lines) in increasing the risk of war onset. A new consensus includes both motivation and opportunity factors.

\begin{itemize}
\setlength\itemsep{0pt}
\item Chps 1 \& 2 in Lars-Erik Cederman, Kristian Skrede Gleditsch \& Halvard Buhaug, \textit{Inequality, grievances, and civil war.} Cambridge UP, 2013: pp. 1--29.
\end{itemize}

\subsection*{Oct 28 - Seminar:}

\begin{itemize}
\setlength\itemsep{-5pt}
\item Laia Balcells, `A way out of Spain's Catalan crisis.' \textit{Foreign Affairs,} 27/11/2019.
\item[] \href{https://www.foreignaffairs.com/articles/europe/2019-11-27/way-out-spains-catalan-crisis}{www.foreignaffairs.com/articles/europe/2019-11-27/way-out-spains-catalan-crisis}
\item[]
\item Lars-Erik Cederman, `Blood for soil: The fatal temptations of ethnic politics.' \textit{Foreign Affairs,} March/April 2019.
\item[] \href{https://www.foreignaffairs.com/articles/2019-02-12/blood-soil}{www.foreignaffairs.com/articles/2019-02-12/blood-soil}
\end{itemize}

% ============================================================
\subsection*{Nov 1 (Lecture) {\color{red}{No class}}}

\subsection*{Nov 4 - Seminar:}

\begin{itemize}
\setlength\itemsep{0pt}
\item Anand Gopal, \textit{No Good Men Among the Living: America, the Taliban, and the War Through Afghan Eyes.} Picador, 2014: pages/chapter TBD.
\end{itemize}

% ============================================================
\subsection*{Nov 8 - Lecture: Violence during war}

The repertoire of violence during wars. Types of violence and definitions. Focus on violence against civilians. Causes and dynamics. Ethnic violence and genocide.

\begin{itemize}
\setlength\itemsep{0pt}
\item Benjamin Valentino, `Why we kill: The political science of political violence against civilians.' \textit{Annual Review of Political Science} 17: 89--103, 2014.
\end{itemize}

\subsection*{Nov 11 - Seminar:}

\begin{itemize}
\setlength\itemsep{0pt}
\item Patrick Radden Keefe, \textit{Say Nothing: A True Story of Murder and Memory in Northern Ireland. }Harper Collins, 2018: pages/chapter TBD.
% \item Chapter 1 in Beatriz Manz, \textit{Paradise in ashes: A Guatemalan journey of courage, terror, and hope.} U of California Press, 2004: pp. 1--32.
\end{itemize}

% ============================================================
\subsection*{Nov 15 - Lecture: Non-state armed groups}

What happens behind the fronts? Rebel governance and recruitment. How do armed groups control the civilian population? Wartime social processes.

\begin{itemize}
\setlength\itemsep{0pt}
\item Chapter 1 (`Introduction') in Ana Arjona, Nelson Kasfir \& Zachariah Mampilly, \textit{Rebel governance in civil war.} Cambridge UP, 2015: pp. 1--20.
\end{itemize}

\subsubsection*{Nov 18 - Seminar:}

\begin{itemize}
\setlength\itemsep{-5pt}
\item Mara Revkin, `ISIS' social contract: What the Islamic State offers civilians.' \textit{Foreign Affairs,} 01/10/2016.
\item[] \href{https://www.foreignaffairs.com/articles/syria/2016-01-10/isis-social-contract}{www.foreignaffairs.com/articles/syria/2016-01-10/isis-social-contract}
\end{itemize}

% ============================================================
\subsection*{Nov 22 - Lecture: Terrorism}

Despite its relevance, terrorism is usually misunderstood. Terrorist actions and terrorist groups. Dynamics and causes. Suicide terrorism.

\begin{itemize}
\setlength\itemsep{0pt}
\item Luis de la Calle \& Ignacio Sánchez-Cuenca, `What we talk about when we talk about terrorism.' \textit{Politics \& Society} 39(3): 451--472, 2011.
\end{itemize}

\subsection*{Nov 25 - Seminar - Reading or Student presentations}

Depending on the number of students. If normal seminar, reading on terrorism:

\begin{itemize}
\setlength\itemsep{0pt}
\item Intro \& chp 1 in Ignacio Sánchez-Cuenca, \textit{The Historical Roots of Political Violence: Revolutionary Terrorism in Affluent Countries.} Cambridge UP, 2019: 1--31.
\end{itemize}

% ============================================================
\subsection*{Nov 29 - Lecture: Legacies of conflict}

Wars, especially wartime violence, transform countries and societies fundamentally. Consequences of civil wars on the civilian population. Long-term legacies on preferences.

\begin{itemize}
\setlength\itemsep{0pt}
\item Elisabeth J Wood, `The social processes of civil war: The wartime transformation of social networks.' \textit{Annual Review of Political Science} 11: 539--561, 2008.
\end{itemize}

\subsection*{Dec 2 - Seminar - Student presentations}

% ============================================================
\subsection*{Dec 6 (Lecture) {\color{red}{No class}}}

\subsection*{Dec 9 - Seminar - Student presentations}

\end{document}
