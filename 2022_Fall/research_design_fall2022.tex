\documentclass[12pt, a4paper]{article}
\usepackage[margin = 2cm]{geometry}
\usepackage{graphicx}
\usepackage[english]{babel}
\usepackage[utf8]{inputenc}
\usepackage[colorlinks = TRUE]{hyperref}
\usepackage{setspace}
\setstretch{1.25}
\renewcommand*\rmdefault{ppl}


\usepackage[]{titlesec}
    \titleformat*{\section}{\large\bf}
    \titleformat*{\subsection}{\normalsize\bf}
    \titleformat*{\subsubsection}{\normalsize\it}

%%%%%%%%%%%%%%%%%%%%%%%%%%%%%%%%%%%%
\begin{document}
\begin{center}
{\LARGE\bf Research Design}\\\vspace{10pt}
Master in Computational Social Science\\
Universidad Carlos III de Madrid\\\vspace{10pt}
{\large Fall 2022}\\
\end{center}

\vspace{20pt}

\begin{minipage}{0.49\textwidth}
\centering
Francisco Villamil\\
\href{francisco.villamil@uc3m.es}{francisco.villamil@uc3m.es}\\
Office: 18.2.D15\\
Office hours: by appointment
\end{minipage}\hfill
\begin{minipage}{0.49\textwidth}
\centering
\textbf{Tuesday 15:00--18:00h}\\Room 0.A.01 (Puerta de Toledo)\\September 13th - October 25th\\
\end{minipage}


\vspace{10pt}
\section{Description}

This course provides an introduction to blah blah

\section{Requirements}

blah blah

Attendance is \textbf{mandatory}.

\section{Materials}

This course does not follow any textbook in particular.
However, there are some books that were used to design the course and can be very useful to expand on what is covered in class:

\begin{itemize}
\setlength\itemsep{-5pt}
  \item Nick Huntington-Klein, \textit{The Effect: An Introduction to Research Design and Causality} (Chapman and Hall/CRC Publishing, 2021).
  \item Ethan Bueno de Mesquita \& Anthony Fowler, \textit{Thinking clearly with data: A guide to quantitative reasoning and analysis} (Princeton University Press, 2021).
  \item Scott Cunningham, \textit{Causal Inference: The Mixtape} (Yale University Press, 2021).
  \item Paul M. Kellstedt \& Guy D. Whitten, \textit{The Fundamentals of Political Science Research} (Cambridge University Press, 2018).
  % \item Philip H. Pollock, \textit{The Essentials of Political Analysis} (Sage, 2016)
  % \item Matthew Salganik ?
\end{itemize}

Some of these books (Huntington-Klein's \textit{The Effect} and Cunningham's \textit{Mixtape}) are available online for free.\footnote{\url{https://theeffectbook.net/} and \url{https://mixtape.scunning.com/}.} Besides, I will make available additional readings in \textit{Aula Global}.

\newpage
\section{Assessment}

% Students will be evaluated based on three different assignments:

\subsection*{Final essay (40\%)}


\subsection*{In-class exercises (30\%)}

In every session, during the second half, we will make practical exercises related to what was covered each day in class. These exercises can be made individually or in groups


\subsection*{Workshop presentation (20\%)}

\begin{itemize}
\setlength\itemsep{-5pt}
  \item Present a topic in general terms, why is it relevant?
  \item Specific research question
  \item Empirical strategy:
  \item[] What type of variation is going to be exploited? Unit of analyses
  \item[] Data: source, variables, measurement
  \item[] Empirical comparisons
\end{itemize}


\subsection*{Participation (10\%)}



\newpage
\section{Course outline}

% \hline % ============================================================
\subsection*{September 13th: Introduction to social research}

\begin{itemize}
\setlength\itemsep{-5pt}
  \item What is social \textit{science}?
  \item Using empirical evidence to answer questions
  \item Importance of research design
  \item Types of research questions and types of empirical research
\end{itemize}

\subsection*{September 20th: Social mechanisms and causality}

\begin{itemize}
\setlength\itemsep{-5pt}
  \item What are we explaining? Building blocks: units, variables, processes
  \item Simplifying social complexity
  \item Different units of analysis and mechanisms
  \item How to approach this empirically? Key idea: variability
\end{itemize}

\subsection*{September 27th: Understanding empirical evidence}

\begin{itemize}
\setlength\itemsep{-5pt}
  \item What is \textit{data}? Types of empirical evidence
  \item Unit of analysis and variability, different approaches %Case studies, comparative studies, quantitative studies with large samples
  \item Raw data and final data, aggregating and disaggregating, measurement problems
\end{itemize}

\subsection*{October 4th: Causal inference design}

\begin{itemize}
\setlength\itemsep{-5pt}
  \item The experimental ideal in the natural and social sciences
  \item How to get closer with observational data?
  \item Confounding, selection bias, collider bias, etc
\end{itemize}

\subsection*{October 11th: Assessing external and internal validity}

\begin{itemize}
\setlength\itemsep{-5pt}
  \item Bringing it all together
  \item Where does evidence come from? External validity
  \item Issues with unit of analysis? Ecological fallacy
  \item Unaccounted for processes? Diffusion, unit independence
\end{itemize}

\subsection*{October 18th: Introduction to causal inference methods}

\begin{itemize}
\setlength\itemsep{-5pt}
  \item Understanding the logic of methods to determine causality, without the use of statistics
  \item Controls, matching
  \item Exploiting exogeneity: difference-in-differences, regression discontuinity
\end{itemize}

\subsection*{October 25th: Presentations workshop}


\subsection*{$*$ Final essay {\color{red}{deadline}}: November 4th, 23.59h}

\end{document}
