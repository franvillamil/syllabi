\documentclass[12pt, a4paper]{article}
\usepackage[margin = 2cm]{geometry}
\usepackage{graphicx}
\usepackage[english]{babel}
\usepackage[utf8]{inputenc}
\usepackage[colorlinks = TRUE]{hyperref}
\usepackage{setspace}
\setstretch{1.25}
\renewcommand*\rmdefault{ppl}


\usepackage[]{titlesec}
    \titleformat*{\section}{\large\bf}
    \titleformat*{\subsection}{\normalsize\bf}
    \titleformat*{\subsubsection}{\normalsize\it}

%%%%%%%%%%%%%%%%%%%%%%%%%%%%%%%%%%%%
\begin{document}
\begin{center}
{\Large War, peace, and political violence}\\\vspace{10pt}
Bachelor in History and Politics\\
Universidad Carlos III de Madrid\\
Fall 2021\\
\end{center}

\vspace{20pt}

\begin{minipage}{0.49\textwidth}
\centering
Francisco Villamil\\
\href{francisco.villamil@uc3m.es}{francisco.villamil@uc3m.es}\\
Office: 18.2.E05\\
Office hours: by appointment
\end{minipage}\hfill
\begin{minipage}{0.49\textwidth}
\centering
Lecture (17.1.04)\\Monday 14:30--16:00h\\\vspace{5pt}
Seminar (14.1.01)\\Thursday 10:45--12:15h\\
\end{minipage}


\vspace{10pt}
\section{Description}

This course provides an overview over a wide range of topics in conflict research, including inter-state wars, civil wars, and causes and dynamics of political violence.
Its main goal is to provide students with the conceptual and theoretical tools to think analytically about conflicts and political violence. Some of the questions we will explore are: Why do countries fight each other? How do changes in the international system impact conflicts across the world? What explains the outbreak of civil wars? Why and how civilians are killed during wars? What are the long-term consequences of conflicts?

\section{Requirements}

We meet twice a week. In the lectures, we will review the main debates in each topic. Each lecture has one reading assigned, usually a research article, that covers part or most of what we will talk about. Reading it is not mandatory, but recommended, either before or after the lecture. In each seminar, we will discuss a reading related to the lecture. These readings, which are \textbf{mandatory}, are shorter and lighter than the ones assigned to the lectures, and are meant to reflect or expand on the topic covered each week.

\section{Materials}

You can find all the reading materials in \textit{Aula Global}.

\newpage
\section{Assessment}

Students will be evaluated based on three different assignments:

\subsection*{Response papers (20\%)}

Each student will have to choose \textbf{two} topics/weeks, and write a 1- to 2-page critical summary of the weekly readings. Response papers (10\% of the grade each) will be due on the seminar day chosen, which can be selected in the Doodle link below (max 3 people per week: first come, first served):

\begin{itemize}
\setlength\itemsep{0pt}
\item[] \href{https://doodle.com/poll/wf3q29s9tzd2expy}{https://doodle.com/poll/wf3q29s9tzd2expy}
\end{itemize}

\subsection*{Presentation (20\%)}

In the last two seminar days (December 2nd and 9th), in groups of 2-3 people, students will have to give a 15-min presentation, which will be followed by a 5- to 10-min Q\&A. Grading will be based on both the presentation and participation during the Q\&A. We can discuss in class alternatives but, in principle, there are two options:

\begin{itemize}
\setlength\itemsep{0pt}
\item[a)] An overview of a single conflict, reflecting on one or more topics covered in class (for example: `Violence against civilians in Syria')
\item[b)] A topic and its relevance in modern or historical times, expanding what we covered in class (for example: `Nationalism and conflict in the 21st Century')
\end{itemize}

\subsection*{Final exam (60\%) -- January 13th, 2020}

Two options for the final exam:

\begin{itemize}
  \item[1.] A final take-home exam. Questions will be handed out at least 24h before deadline. Its goal is to evaluate how well students understood the main concepts and ideas.
  \item[2.]
\end{itemize}

\newpage
\section{Course outline}

\subsection*{Lecture (6 Sept) - Introduction}

Presentation. Course structure and organizational issues. Introduction: what is political violence and what are we going to talk about?

\begin{itemize}
\setlength\itemsep{-5pt}
\item \textit{No readings}
\end{itemize}

\subsection*{Seminar (9 Sept) {\color{red}{(No class)}}}

% ============================================================
\subsection*{Lecture (13 Sept) - Concepts}

\subsection*{Seminar (16 Sept)}

% ============================================================
\subsection*{Lecture (20 Sept) {\color{red}{(No class)}}}

\textit{No class.}

\subsection*{Seminar (23 Sept)}

% ============================================================
\subsection*{Lecture (27 Sept) - xxxxx}

\subsection*{Seminar (30 Sept)}

% ============================================================
\subsection*{Lecture (4 Oct) - xxxxx}

\subsection*{Seminar (7 Oct)}

% ============================================================
\subsection*{Lecture (11 Oct) {\color{red}{(No class)}}}

\textit{No class.}

\subsection*{Seminar (14 Oct)}

% ============================================================
\subsection*{Lecture (18 Oct) - xxxxx}

\subsection*{Seminar (21 Oct)}

% ============================================================
\subsection*{Lecture (25 Oct) - xxxxx}

\subsection*{Seminar (28 Oct)}

% ============================================================
\subsection*{Lecture (1 Nov) {\color{red}{(No class)}}}

\textit{No class.}

\subsection*{Seminar (4 Nov)}

% ============================================================
\subsection*{Lecture (8 Nov) - xxxxx}

\subsection*{Seminar (11 Nov)}

% ============================================================
\subsection*{Lecture (15 Nov) - xxxxx}

\subsection*{Seminar (18 Nov)}

% ============================================================
\subsection*{Lecture (22 Nov) - xxxxx}

\subsection*{Seminar (25 Nov)}

% ============================================================
\subsection*{Lecture (29 Nov) - xxxxx}

\subsection*{Seminar (2 Dec) - Presentations 1st day}

% ============================================================
\subsection*{Lecture (6 Dec) {\color{red}{(No class)}}}

\textit{No class.}

\subsection*{Seminar (9 Dec) - Presentations 2nd day}

\end{document}
