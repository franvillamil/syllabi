\documentclass[12pt, a4paper]{article}
\usepackage[margin = 2cm]{geometry}
\usepackage{graphicx}
\usepackage[english]{babel}
\usepackage[utf8]{inputenc}

\usepackage{color}
	\definecolor{dark_blue}{rgb}{0.2, 0.0, .7}
\usepackage[colorlinks = TRUE,
			allcolors = dark_blue]{hyperref}

\usepackage{setspace}
\setstretch{1.25}
\renewcommand*\rmdefault{ppl}


\usepackage[]{titlesec}
    \titleformat*{\section}{\large\bf}
    \titleformat*{\subsection}{\normalsize\bf}
    \titleformat*{\subsubsection}{\normalsize\it}

%%%%%%%%%%%%%%%%%%%%%%%%%%%%%%%%%%%%
\begin{document}
\begin{center}
{\LARGE\bf International Relations}\\\vspace{10pt}
Master in Social Sciences\\
Carlos III-Juan March Institute\\Universidad Carlos III de Madrid\\\vspace{10pt}
{\large Spring 2023}\\
\end{center}

\vspace{20pt}

\begin{minipage}{0.49\textwidth}
% \centering
\textbf{Francisco Villamil}\\
Email: \href{francisco.villamil@uc3m.es}{francisco.villamil@uc3m.es}\\
Office: 18.2.D15\\Office hours: Thu 15-16h or by appointment\\
\end{minipage}\hfill
\begin{minipage}{0.49\textwidth}
\centering
\textbf{Time and place:}\\
Thursdays, 10h--13h\\Room 18.1.A01\\
\end{minipage}


\vspace{10pt}
\section{Description}

This is a graduate-level course on contemporary research on international relations. The core of the field of international relations focuses on the causes, dynamics, and consequences of aggression and cooperation between states, and has traditionally revolved around the study of war and peace.
We will focus on the study of international conflict, but also on related topics that are often considered within IR, such as the study of internal conflict and processes of political violence, the nature of state size and borders, the influence of the international system on domestic institutions, or processes of international diffusion of norms.
Although, especially during the first sessions, we will review some classic works that have a strong theoretical component, the focus of this course will be on more recent research and on the methodological aspect of studying these topics empirically.

\section{Requirements and assessment}

Each session will consist of a brief lecture introducing the topic and the main concepts, followed by a seminar discussion on the required readings. Students are expected to attend every week, read the required readings in advance, and actively participate in the discussion.
In addition, each student will have to present one of the complementary readings once during the semester.
The last session of the course will be a workshop where students present ideas and ongoing work for the final research paper.

\paragraph{Participation (20\%):}

blah blah blah

\paragraph{Presentation (20\%):}

blah blah blah

\paragraph{Research paper (60\%):}

blah blah blah

% \newpage
\section{Course outline}

 % \hline % ============================================================

\subsection*{Session 1: Introduction}

\noindent\textit{Required:}

\begin{itemize}
  \item Stephen M. Walt (1998) \href{https://doi.org/10.2307/1149275}{International Relations: One World, Many Theories.} \textit{Foreign Policy} 110: 29--32, 34--46.
  \item David A Lake (2011) \href{https://academic.oup.com/isq/article/55/2/465/1796473}{Why `isms' Are Evil: Theory, Epistemology, and Academic Sects as Impediments to Understanding and Progress.} \textit{International Studies Quarterly} 55(2): 465--480.
  \item Bruce Bueno de Mesquita (1985) \href{https://doi.org/10.2307/2600500}{Toward a Scientific Understanding of International Conflict: A Personal View.} \textit{International Studies Quarterly} 29(2): 121--136.
\end{itemize}

% - DOMESTIC POLITICS, FOREIGN POLICY, AND THEORIES OF INTERNATIONAL RELATIONS
% - Sikking on constructivism

\noindent\textit{Complementary:}

\begin{itemize}
  \item (No complementary readings)
\end{itemize}

\hline
% ------------------------------------------------------------
\subsection*{Session 2: Interstate war}

\noindent\textit{Required:}

\begin{itemize}
  \item Kenneth Waltz, \textit{Man, the State and War.} Columbia University Press, 1959. Chapters 2, 4, and 6.
  \item
  \item James D. Fearon (1995) \href{https://doi.org/10.1017/S0020818300033324}{Rationalist Explanations for War.} \textit{International Organization} 49(3): 379--414.
  \item Paul Huth, Christopher Gelpi, and D. Scott Bennett (1993) \href{https://doi.org/10.2307/2938739}{The Escalation of Great Power Militarized Disputes: Testing Rational Deterrence Theory and Structural Realism.} \textit{American Political Science Review} 87(3): 609--623.
\end{itemize}




\noindent\textit{Complementary:}

\begin{itemize}
  \item Kalevi J Holsti, \textit{Peace and War: Armed Conflicts and International Order, 1648-1989.} Cambridge University Press, 1991. Chapter 1: pp. 1--24.
  \item Powell, Robert (1996) \href{https://doi.org/10.2307/2945840}{Uncertainty, Shifting Power, and Appeasement.} \textit{American Political Science Review} 90(4): 749--764.
  \item James D. Fearon (1994) \href{https://doi.org/10.2307/2944796}{Domestic Political Audiences and the Escalation of International Disputes.} \textit{American Political Science Review} 88(3): 577--592.
\end{itemize}

\hline
% ------------------------------------------------------------
\subsection*{Session 3: State-building, nationalism, and war}

\noindent\textit{Required:}

\begin{itemize}
  \item Charles Tilly, `War making and state making as organized crime.' In \textit{Bringing the State Back In} (ed. P. Evans, D. Rueschemyer \& T. Skocpol). Cambridge UP, 1985: pp. 169--187.
  \item Andreas Wimmer and Brian Min, `From empire to nation-state. Explaining wars in the modern world, 1816-2001.' \textit{American Sociological Review} 71: 867--897, 2006.
  \item Lars-Erik Cederman, T. Camber Warren \& Didier Sornette, `Testing Clausewitz: Nationalism, Mass Mobilization, and the Severity of War.' \textit{International Organization} 65(4): 605--638, 2011.
  \item Agustina S. Paglayan (2022) \href{https://doi.org/10.1017/S0003055422000247}{Education or Indoctrination? The Violent Origins of Public School Systems in an Era of State-Building.} \textit{American Political Science Review} 116(4): 1242--1257
\end{itemize}

\noindent\textit{Complementary:}

\begin{itemize}
  \item
  \item
\end{itemize}

\hline
% ------------------------------------------------------------
\subsection*{Session 4: Peace and democracies}

% Oneal, John, Bruce Russett, and Michael Berbaum. 2003. Causes of Peace: Democracy,
% Interdependence, and International Organizations, 1885-1992. International Studies Quarterly 47 (3): 371-394.
% Dixon, William. 1994. Democracy and the Peaceful Settlement of Conflicts. American Political Science Review 88 (1): 14-32.
% Pevehouse, Jon, and Bruce Russett. 2006. Democratic International Governmental Organizations Promote Peace. International Organization 60 (4): 969-1000.
% Layne, Christopher. 1994. Kant or Cant: The Myth of the Democratic Peace. International Security 19 (2): 5-49.

% Mitchell, Sara McLaughlin. 2002. A Kantian System? Democracy and Third Party Conflict Resolution. American Journal of Political Science 46 (4): 749-759.


\noindent\textit{Required:}

\begin{itemize}
  \item
  \item
  \item Cederman, Lars-Erik (2001) \href{https://doi.org/10.1017/S0003055401000028}{Back to Kant: Reinterpreting the Democratic Peace as a Macrohistorical Learning Process.} \textit{American Political Science Review} 95(1): 15--31.
\end{itemize}

\noindent\textit{Complementary:}

\begin{itemize}
  \item
  \item
\end{itemize}

\hline
% ------------------------------------------------------------
\subsection*{Session 5: Political economy}

\noindent\textit{Required:}

\begin{itemize}
  \item Rodrik, Dani (2021) \href{https://www.annualreviews.org/doi/abs/10.1146/annurev-economics-070220-032416}{Why Does Globalization Fuel Populism? Economics, Culture, and the Rise of Right-Wing Populism.} \textit{Annual Review of Economics} 13: 133--170.
  \item
  \item
\end{itemize}

\noindent\textit{Complementary:}

\begin{itemize}
  \item
  \item
\end{itemize}

\hline
% ------------------------------------------------------------
\subsection*{Session 6: International borders and state size}

meter aqui algo de genocide? intl approach

\noindent\textit{Required:}

\begin{itemize}
  \item
  \item
  \item
\end{itemize}

\noindent\textit{Complementary:}

\begin{itemize}
  \item
  \item
\end{itemize}

\hline
% ------------------------------------------------------------
\subsection*{Session 7: Civil wars}

\noindent\textit{Required:}

\begin{itemize}
  \item James D. Fearon and David Laitin (2003) \href{https://doi.org/10.1017/S0003055403000534}{Ethnicity, Insurgency, and Civil War} \textit{American Political Science Review} 97(1): 75--90.
  \item Lars-Erik Cederman, Andreas Wimmer, and Brian Min (2010) \href{https://doi.org/10.1017/S0043887109990219}{Why Do Ethnic Groups Rebel? New Data and Analysis.} \textit{World Politics} 62(1): 87--119.
  \item Michael Doyle and Nicholas Sambanis (2000) \href{https://doi.org/10.2307/2586208}{International Peacebuilding: A Theoretical and Quantitative Analysis.} \textit{American Political Science Review} 94(4): 779--802.
\end{itemize}

\noindent\textit{Complementary:}

\begin{itemize}
  \item \item Lars-Erik Cederman, Kristian Skrede Gleditsch, and Nils B. Weidmann (2011) \href{https://doi.org/10.1017/S0003055411000207}{Horizontal Inequalities and Ethnonationalist Civil War: A Global Comparison.} \textit{American Political Science Review} 105(3): 478--495.
  % \item Lars-Erik Cederman and Luc Girardin (2007) \href{https://doi.org/10.1017/S0003055407070086}{Beyond Fractionalization: Mapping Ethnicity onto Nationalist Insurgencies.} \textit{American Political Science Review} 101(1): 173--185.
  \item
\end{itemize}

\hline
% ------------------------------------------------------------
\subsection*{Session 8: Wartime violence}

\noindent\textit{Required:}

\begin{itemize}
  \item
  \item
  \item
\end{itemize}

\noindent\textit{Complementary:}

\begin{itemize}
  \item Robert Braun (2016) \href{https://doi.org/10.1017/S0003055415000544}{Religious Minorities and Resistance to Genocide: The Collective Rescue of Jews in the Netherlands during the Holocaust.} \textit{American Political Science Review} 110(1): 127--147.
  \item
\end{itemize}


\hline
% ------------------------------------------------------------
\subsection*{Session 9: Terrorism and armed groups}

\noindent\textit{Required:}


Kydd, Andrew. H, and Walter, Barbara. F. 2006. “The Strategies of Terrorism.” International Security 31(1): 49-80.
David Lake. 2002. “Rational Extremism: Understanding Terrorism in the Twenty First Century.” International Organization 56(1): 15-29.
Savun, Burcu and Brian J. Phillips. 2009. “Democracy, Foreign Policy, and Terrorism,” Journal of Conflict Resolution 53(6): 878-904.


Pape, Robert. 2003. “The Strategic Logic of Suicide Terrorism.” American Political Science Review 97(3): 343-361.


\begin{itemize}
  \item
  \item
  \item
\end{itemize}

\noindent\textit{Complementary:}

\begin{itemize}
  \item
  \item
\end{itemize}

\hline
% ------------------------------------------------------------
\subsection*{Session 10: Legacies of conflict}

\noindent\textit{Required:}

\begin{itemize}
  \item
  \item
  \item
\end{itemize}

\noindent\textit{Complementary:}

\begin{itemize}
  \item
  \item
\end{itemize}

\hline
% ------------------------------------------------------------
\subsection*{Session 11: Transitional justice and norm diffusion}

aprovechar y meter rollos de constructivism?

\noindent\textit{Required:}

\begin{itemize}
  \item Kathryn Sikkink and Carrie Booth Walling (2007) \href{https://doi.org/10.1177/002234330707895}{The Impact of Human Rights Trials in Latin America.} \textit{Journal of Peace Research} 44(4): 427--445.
  \item {\color{red}{Kim HJ. 2012. Structural determinants of human rights prosecutions after democratic transition. J. Peace Res. 49(2):305–20}}
  \item {\color{red}{Kim HJ, Sikkink K. 2010. Explaining the deterrence effect of human rights prosecutions for transitional countries. Int. Stud. Q. 54:939–63}}
\end{itemize}

Wendt, Alexander. 1992. Anarchy Is What States Make Of It: The Social Construction of Power
Politics. International Organization 46 (4): 887-918.
Finnemore, Martha, and Kathryn Sikkink. 1998. International Norm Dynamics and Political
Change. International Organization 52 (4): 887-918.
Adler, Emmanuel. 1997. “Seizing the Middle Ground: Constructivism in World Politics.”
European Journal of International Relations 3, 3: 319-63
Ruggie, John Gerard. 1986. Continuity and Transformation in the World Polity: Toward a
Neorealist Synthesis. In Robert Keohane, Ed. Neorealism and Its Critics. New York: Columbia University Press: 131-157.
Katzenstein, Peter. 2003. Same War-Different Views: Germany, Japan, and Counterterrorism.
International Organization 57 (4): 731-760.
Lebovic, James, and Erik Voeten. 2006. The Politics of Shame: The Condemnation of Country
Human Rights Practices in the UNHCR. International Studies Quarterly 50 (4): 861-888.


\noindent\textit{Complementary:}

\begin{itemize}
  \item
  \item
\end{itemize}

\hline
% ------------------------------------------------------------
\subsection*{Session 12: Research workshop}

No readings. Presentation of research projects and workshop discussion.

\end{document}
