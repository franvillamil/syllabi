\documentclass[12pt, a4paper]{article}
\usepackage[margin = 2cm]{geometry}
\usepackage{graphicx, booktabs}
\usepackage[english]{babel}
\usepackage[utf8]{inputenc}
\usepackage{color}
\definecolor{dark_blue}{rgb}{0.2, 0.0, .7}
\usepackage[colorlinks = TRUE, linkcolor = dark_blue]{hyperref}
\usepackage{setspace}
\setstretch{1.25}
\renewcommand*\rmdefault{ppl}

\usepackage[]{titlesec}
    \titleformat*{\section}{\large\bf}
    \titleformat*{\subsection}{\normalsize\bf}
    \titleformat*{\subsubsection}{\normalsize\it}

%%%%%%%%%%%%%%%%%%%%%%%%%%%%%%%%%%%%
\begin{document}
\begin{center}
{\LARGE\bfseries Política Mundial}\\\vspace{10pt}
Grupos 88 y 188 (CCPP y Sociología)\\
Universidad Carlos III de Madrid\\\vspace{10pt}
{\large\bfseries Segundo cuatrimestre 2025--2026}\\\vspace{15pt}
Web: \href{https://franvillamil.github.io/pol_mundial/}{franvillamil.github.io/pol\_mundial/}
\end{center}

\section{Información y horarios}

\begin{minipage}{0.49\textwidth}
\flushleft
\textbf{Profesor magistrales:}\\
Francisco Villamil\\
\href{francisco.villamil@uc3m.es}{francisco.villamil@uc3m.es}\\
Despacho: 18.2.A34\\
Tutorías: Martes 14h-16h (cita por mail)
\end{minipage}\hfill
\begin{minipage}{0.49\textwidth}
\flushleft
\textbf{Profesor seminarios:}\\
Luis Azores\\
\href{lazores@pa.uc3m.es}{lazores@pa.uc3m.es}\\
Despacho:\\
Tutorías: cita por email
\end{minipage}

\vspace{15pt}

\begin{center}
\begin{tabular}{|l|ll|cc|}
\hline
& \textbf{Magistral} & \textbf{Seminario} & \textbf{Examen} (ord) & \textbf{Examen} (ext) \\
\hline
\textbf{Grupo 88} & Martes, 9:00-10:30 & Viernes 10:45-12:15 & 21 Mayo & 18 Junio \\
& Aula 9.2.03 & Aula 15.0.16 & (18h-21h) & (9h-12h) \\
\hline
\textbf{Grupo 188} & Martes, 16:15-17:45 & Jueves 18:00-19.30 & 28 Mayo & 18 Junio \\
& Aula 4.1.03 & Aula 4.1.06 & (12h-15h) & (9h-12h) \\
\hline
\end{tabular}
\end{center}


\vspace{10pt}
\section{Descripción}

Esta asignatura ofrece una introducción al estudio de la política mundial contemporánea. A lo largo del curso se analizan las principales dinámicas que han configurado el sistema internacional, desde la formación del Estado moderno hasta los desafíos actuales. El temario aborda cuestiones fundamentales como las causas de la guerra y la violencia organizada, el terrorismo, el papel de las normas internacionales y los derechos humanos, el comercio internacional, las finanzas globales y el desarrollo económico. El objetivo es que los estudiantes adquieran herramientas analíticas para comprender las interacciones entre Estados y otros actores internacionales, prestando atención a los intereses, las instituciones y las dinámicas de cooperación y conflicto que definen la política global.

\clearpage
\section{Requisitos generales y recursos}

\begin{itemize}
    \item[] \textbf{Libro de texto principal:}
    \item Jeffry A. Frieden, David A. Lake, and Kenneth A. Schultz, \textit{World Politics: Interests, Interactions, Institutions.} Ed: W.W. Norton (actualmente usando la 4ta edición, 2019).
    \item[] \textbf{Lecturas complementarias:}
    \item Se irán añadiendo a la página web (\href{https://franvillamil.github.io/pol_mundial/resources.html}{franvillamil.github.io/pol\_mundial/resources.html}) así como a Aula Global, tanto las de apoyo a las clases magistrales como las que se utilicen para los seminarios.
\end{itemize}

% \newpage
\section{Evaluación}

\begin{itemize}
\setlength\itemsep{0pt}
  \item \textbf{20\%}: Actividades seminarios
  \item \textbf{30\%}: Presentación
  \item \textbf{50\%}: Examen final
\end{itemize}

\subsection*{Actividades seminarios (20\%)}

Durante los seminarios se harán varias actividades en las cuales los estudiantes tendrán que hacer alguna entrega. Estas entregas se corregirán y servirán para asignar esta parte de la nota, que mide tanto la asistencia como el compromiso activo.

\subsection*{Presentación (30\%)}

Durante 2-3 sesiones al final del curso se realizarán presentaciones en grupo de un tema relacionado con el temario de la asignatura. Éstas se realizarán en los seminarios pero se reservará una clase magistral en caso de que no de tiempo en los seminarios o alguno de ellos sea un día festivo. Los grupos serán de 3-4 personas. Se pide a los estudiantes que confirmen con el profesor de las clases reducidas el tema de las presentaciones. Se comentarán los detalles sobre ellas en los seminarios reducidos.

Para el curso 2025--2026 se reservan 3 sesiones para hacer estas presentaciones. Los dos primeras tendrán lugar con seguridad y la tercera se programa para en caso de que haya un mayor número de grupos. Fechas:

\begin{center}
\begin{tabular}{|c|c|c|}
\hline
& \textbf{Grupo 88} & \textbf{Grupo 188} \\
\hline
1er día: & Jueves 23/04 & Viernes 24/04 \\
2do día: & Jueves 30/05 & Martes 05/05 \\
3er día: & Martes 05/05 & Viernes 08/05 \\
\hline
\end{tabular}
\end{center}


\subsection*{Examen final (50\%)}

El examen consistirá en dos preguntas a desarrollar que tendrán que escribirse a mano en clase. El examen tendrá una duración de 1 hora. Cada una de las preguntas tendrá un límite corto de extensión definido, para priorizar que se planteen de antemano como un argumento corto y razonado.

Se podrán utilizar libros, apuntes o documentos durante su realización, siempre y cuando estos estén en papel. \textbf{No se permitirá el uso de dispositivos electrónicos} durante el examen, y cualquier detección de plagio o uso de AI conlleva un suspenso automático. %Al final del examen, se elegirá aleatoriamente a dos estudiantes para que explican oralmente sus respuestas.

\subsection*{Examen parcial}

A mitad del cuatrimestre, se realizará un examen parcial cubriendo la primera mitad del temario. Este examen tendrá el mismo formato que el examen final, pero sólo incluirá una pregunta. Los estudiantes podrán elegir eliminar materia en el examen final y quedarse con la nota del examen parcial para la primera pregunta del examen final. Esta elección tendrá que hacerse \textit{antes} de hacer el examen final, no será posible escoger después de saber las notas del examen final.

\section{Programa}

(Resumen y calendario disponible en la web del curso: \href{https://franvillamil.github.io/pol_mundial/}{franvillamil.github.io/pol\_mundial/})

\begin{center}
\textbf{Contenidos y temas}\\\vspace{10pt}
\begin{tabular}{|ll|}
\hline
\textit{\small Introducción} & El mundo hoy \\
  & Actores e intereses \\
\hline
\textit{\small Guerra y violencia} & El problema de la guerra \\
  & Guerras y organizaciones \\
  & Terrorismo \\
\hline
\textit{\small Normas} & Leyes, normas y DDHH \\
\hline
\textit{\small Economía política internacional} & Comercio internacional \\
  & Finanzas y dinero \\
  & Desarrollo económico \\
\hline
\textit{\small General} & Futuro y problemas emergentes \\
\hline
\end{tabular}
\end{center}

\end{document}
